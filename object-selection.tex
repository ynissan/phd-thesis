\clearpage
\section{Object definition and selection}
\label{sec:object-selection}
We have seen in~\ref{sec:reconstruction-and-identification} how objects are being reconstructed and identified in our detector. We have also studied the signal signature in~\ref{sec:signal-signature}. In this section we devise an object selection in order to obtain as pure as possible sample of objects in regards to our target leptons, while retaining as much signal as possible. As we have seen in~\ref{sec:search-strategy}, we are targeting the opposite-charged same-flavor leptons \ellell that result from the \neutt that decays into a \neuto via a \PZstar, \ie, \neuttdecay. In the following section, we choose to present two choices of \gls{dmo}, namely, $\dmo=1.92\GeV$ and $\dmo=5.63\GeV$, \ie, a relatively high \gls{dmo}, and a low one, but not too low as to still be able to have enough electrons surviving the initial reconstruction \pt threshold of $5\GeV$. We also fix the higgsino parameter on $\mu=100\GeV$.

As was the case in~\ref{sec:signal-signature}, the base selection for the following section is requiring at least one jet in the event with $\pt \geq 30\GeV$ and $\abs{\eta}<2.4$. No other selections otherwise. However, unlike~\ref{sec:signal-signature}, we do not weight our objects to any luminosity, as we are interested in the proportion between object types. We differentiate between two types of leptons, ones that originate from our targeted decay \neuttdecay, which will be shown in blue, and those that do not, which we refer to as \emph{other}, and are shown in yellow. Leptons that are marked as resulting from the \neuttdecay decay, which we will refer to as \emph{signal leptons}, are done so by matching a reconstructed lepton to a generator level lepton, which has been checked to have the \neutt as its parent. Lepton marked as \emph{other}, either has been misreconstructed, misidentified or is a result of hadronisation process in a jet (such as the \gls{isr} jet). Our goal here is to select as many blue leptons as possible, while rejecting as many yellow ones as possible. In the following sections, we will refer to \emph{efficiency} as the proportion between the signal leptons passing a selection, divided by the initial number of signal leptons, and to \emph{purity} as the proportion between signal leptons (blue) and the sum of the signal leptons and \emph{other} leptons (yellow). So to rephrase our goal, we are interested in a selection that results in high-efficiency and high-purity. These two quantities can sometimes compete with each other and we have to make compromises.

\subsection{Electrons}
\label{sec:object-selection-electrons}
%\subsubsection{Signal electron selection}

The electrons have an initial reconstruction \pt threshold of $5\GeV$. The initial working point choice for reconstructed electron is loose (see~\ref{sec:reconstruction-and-identification}). The first distribution we look at in regards to the electrons is their spatial separation from the leading jet in the event, $\DR(\jmath_1,\Pe)$. Those can be seen in~\ref{fig:electrons-dr-lj}.

\begin{figure}[h]
\centering
\includegraphics[width=0.48\linewidth]{plots/lepton_selection/lepton_selection_dm5p63/none_Electrons_rlj.pdf} \,
\includegraphics[width=0.48\linewidth]{plots/lepton_selection/lepton_selection_dm1p92/none_Electrons_rlj.pdf}  \\
\caption[Spatial seperation between reconstructed electrons and the leading jet $\DR(\jmath_1,\Pe)$]{Spatial seperation between reconstructed electrons with loose ID and the leading jet $\DR(\jmath_1,\Pe)$ for $\dm=5.63\GeV$ (left) and $\dm=1.92\GeV$ (right).}
\label{fig:electrons-dr-lj}
\end{figure}

There are two obvious features we can point to in these plots. We have explored the first feature already in~\ref{sec:signal-signature}, namely, that probing lower \gls{dm} requires access to low \gls{pt} leptons, and since we are limited by a lower threshold of $\pt\geq 5\GeV$ on the electrons, that results in lower signal acceptance as can be seen by the difference between the high \gls{dm} and the low one. The second interesting feature that we can see, is that our signal electrons are located mainly outside of the leading jet. That is because the leading jet is usually an \gls{isr} jet which boosts the \tchiz system away from it (back-to-back). We therefore make a cut $\DR(\jmath_1,\Pe)>0.4$.

Next we turn to the \gls{pt} distributions. First we apply the previous cut of $\DR(\jmath_1,\Pe)>0.4$. As we've already seen in~\ref{sec:muon-eta-pt}, the \gls{pt} distribution depends strongly on \gls{dm}. Even though the distributions in~\ref{sec:muon-eta-pt} were plotted using generator level muons, the electrons distributions follow the same trend. We therefore need to make a choice about which \gls{dm} to favor, \ie, which \gls{dm} we want to be more sensitive to, and we choose the lower \gls{dm} case. Nonetheless we compare the two choices in~\ref{fig:electrons-selection-pt}.

\begin{figure}[!htb]
\centering
\includegraphics[width=0.48\linewidth]{plots/lepton_selection/lepton_selection_dm5p63/none_Electrons_pt.pdf} \,
\includegraphics[width=0.48\linewidth]{plots/lepton_selection/lepton_selection_dm1p92/none_Electrons_pt.pdf}  \\
\caption[\pt distribution of reconstructed electrons with loose ID]{ \pt distribution of reconstructed electrons with loose ID for $\dm=5.63\GeV$ (left) and $\dm=1.92\GeV$ (right). Cut of $\DR(\jmath_1,\Pe)>0.4$ applied.}
\label{fig:electrons-selection-pt}
\end{figure}

We can see, as expected, that the \pt distribution  of the electrons fall more rapidly for the low \dm case. We observe that there are hardly any electrons surviving above $15\GeV$, and therefore we choose to make a cut of $\pt<15\GeV$.

It interesting to look at the $\eta$ distribution, as seen in~\ref{fig:electrons-selection-eta}, after the previous cuts to get a better sense of where most of the non-signal electrons are still coming from.

\begin{figure}[!htb]
\centering
\includegraphics[width=0.48\linewidth]{plots/lepton_selection/lepton_selection_dm5p63/none_Electrons_eta.pdf} \,
\includegraphics[width=0.48\linewidth]{plots/lepton_selection/lepton_selection_dm1p92/none_Electrons_eta.pdf}  \\
\caption[\abs{\eta} distribution of reconstructed electrons with loose ID]{ \abs{\eta} distribution of reconstructed electrons with loose ID for $\dm=5.63\GeV$ (left) and $\dm=1.92\GeV$ (right). Cuts of $\DR(\jmath_1,\Pe)>0.4$ and $\pt<15\GeV$ are applied.}
\label{fig:electrons-selection-eta}
\end{figure}

In the case of $\dm=1.92\GeV$,  we can clearly see how worse the endcaps of the \gls{ecal} are performing in comparison with the barrel ($\abs{\eta}<1.48$). The transition is clearly visible through a sharp drop in purity at the transition. It is worse for low-\pt electrons than higher-\pt ones.

We would like to see if requiring a tighter working point for the electron-identification is beneficial. The working point used in the previous distributions is loose. We turn now to check the effects of requiring either a medium working point, or a tight one. We plot two bins labeled \emph{fail} and \emph{pass}, which correspond to whether the electron passes or fails the identification criteria of a medium or tight working points. These can be seen in~\ref{fig:electrons-selection-id}.

\begin{figure}[!htb]
\centering
\includegraphics[width=0.48\linewidth]{plots/lepton_selection/lepton_selection_dm5p63/none_Electrons_medium.pdf} \,
\includegraphics[width=0.48\linewidth]{plots/lepton_selection/lepton_selection_dm1p92/none_Electrons_medium.pdf}  \\
\includegraphics[width=0.48\linewidth]{plots/lepton_selection/lepton_selection_dm5p63/none_Electrons_tight.pdf} \,
\includegraphics[width=0.48\linewidth]{plots/lepton_selection/lepton_selection_dm1p92/none_Electrons_tight.pdf}  \\
\caption[medium and tight ID working points distribution of reconstructed electrons]{Medium (top) and tight (bottom) ID working points distributions of reconstructed electrons for $\dm=5.63\GeV$ (left) and $\dm=1.92\GeV$ (right). Cuts of $\DR(\jmath_1,\Pe)>0.4$ and $\pt<15\GeV$ are applied.}
\label{fig:electrons-selection-id}
\end{figure}

Selecting a medium or a tight working point is equivalent to choosing the relevant right \emph{pass} bin (top for medium, bottom for tight), and rejecting the electrons on the left \emph{fail} bin. We see that although we reject considerable amount of non-signal electrons in the low \dm case by picking either a medium or tight working points, we also loose quite a lot of signal electrons as well. In other words, these selections are very not efficient and will result in low signal acceptance. We therefore decide to use a loose working point for the electrons. We will see that we can still purify the electron selection by relying on isolation instead. We fully discuss and describe our jet-isolation in~\ref{sec:isolation}, but here for the sake of completeness we look at its effect on the purity of the electrons. We compare our custom jet-isolation to the standard definition of lepton isolation, which does not take into account the possibility that two electrons can be produced close to each other (small \DR), as is the case in our signal. Comparison of the isolation distributions are seen in~\ref{fig:electrons-selection-isolation}.

\begin{figure}[!htb]
\centering
\includegraphics[width=0.48\linewidth]{plots/lepton_selection/lepton_selection_dm5p63/none_Electrons_iso.pdf} \,
\includegraphics[width=0.48\linewidth]{plots/lepton_selection/lepton_selection_dm1p92/none_Electrons_iso.pdf}  \\
\includegraphics[width=0.48\linewidth]{plots/lepton_selection/lepton_selection_dm5p63/none_Electrons_CorrJetNoMultIso11Dr0.5.pdf} \,
\includegraphics[width=0.48\linewidth]{plots/lepton_selection/lepton_selection_dm1p92/none_Electrons_CorrJetNoMultIso11Dr0.5.pdf}  \\
\caption[standard isolation and jet-isolation distribution of reconstructed electrons]{Standard isolation (top) and custom jet-isolation (bottom) distributions of reconstructed electrons with loose ID for $\dm=5.63\GeV$ (left) and $\dm=1.92\GeV$ (right). Cuts of $\DR(\jmath_1,\Pe)>0.4$ and $\pt<15\GeV$ are applied.}
\label{fig:electrons-selection-isolation}
\end{figure}

We observe that the standard lepton isolation does not perform well in terms of efficiency for both \dm cases. In contrast, the custom jet-isolation is performing very well in terms of signal electron efficiency while successfully rejecting considerable amount of non-signal electrons, resulting in a purer sample of electrons. We therefore conclude that the choice of the custom jet-isolation is favorable. We also look at how the $\eta$ distribution is affected by this choice at~\ref{fig:electrons-selection-eta-jet-iso}, and that will conclude our selection of the electrons.

\begin{figure}[!htb]
\centering
\includegraphics[width=0.48\linewidth]{plots/lepton_selection/lepton_selection_dm5p63/none_Electrons_eta_jet_iso.pdf} \,
\includegraphics[width=0.48\linewidth]{plots/lepton_selection/lepton_selection_dm1p92/none_Electrons_eta_jet_iso.pdf}  \\
\caption[\abs{\eta} distribution of reconstructed electrons with loose ID passing jet-isolation]{ \abs{\eta} distribution of reconstructed electrons with loose ID passing jet-isolation for $\dm=5.63\GeV$ (left) and $\dm=1.92\GeV$ (right). Cuts of $\DR(\jmath_1,\Pe)>0.4$ and $\pt<15\GeV$ are applied.}
\label{fig:electrons-selection-eta-jet-iso}
\end{figure}

When we compare distributions~\ref{fig:electrons-selection-eta-jet-iso} with~\ref{fig:electrons-selection-eta}, we can see that our custom jet-isolation has done a good job purifying the electrons selection even further while being efficient in retaining the signal electrons.

%\subsubsection{Selection summary}

We summaries this section with the full selection of the analysis electrons:

\begin{itemize}
\item $5<\pt<15\GeV$
\item $\abs{\eta} < 2.5$
\item $\DR(\jmath_1,\Pe)>0.4$
\item loose ID working point
\item pass jet-isolation
\end{itemize}

\clearpage

\subsection{Muons}
\label{sec:muon-selection}

In contrast to the electrons, we do not have an initial reconstruction \pt threshold of $5\GeV$. Therefore, we want to explore the possibility of lowering the \pt threshold as much as posible. This has been motivated in section~\ref{sec:muon-eta-pt}, where we saw that the lower \dm we want to probe, the lower \pt threshold we have to allow. Like in the electron case, the initial working point choice for reconstructed muon is loose (see~\ref{sec:reconstruction-and-identification}). We follow a similar procedure to the electrons case. The first distribution we look at in regards to the muons is their spatial separation from the leading jet in the event, $\DR(\jmath_1,\mu)$. We have seen in~\ref{fig:signal-pt-barrel-endcaps} that the muon endcaps are capable of reconstructing muons with $\pt<3\GeV$ while the barrel cannot. It therefore makes sense to look at a split view of barrel and endcaps for the following distributions at~\ref{fig:muons-dr-lj}.

\begin{figure}[!htb]
\centering
\includegraphics[width=0.32\linewidth]{plots/lepton_selection/lepton_selection_dm5p63/none_Muons_rlj.pdf} \,
\includegraphics[width=0.32\linewidth]{plots/lepton_selection/lepton_selection_dm5p63/none_Muons_rlj_barrel.pdf}
\includegraphics[width=0.32\linewidth]{plots/lepton_selection/lepton_selection_dm5p63/none_Muons_rlj_endcape.pdf}  \\
\includegraphics[width=0.32\linewidth]{plots/lepton_selection/lepton_selection_dm1p92/none_Muons_rlj.pdf} \,
\includegraphics[width=0.32\linewidth]{plots/lepton_selection/lepton_selection_dm1p92/none_Muons_rlj_barrel.pdf}
\includegraphics[width=0.32\linewidth]{plots/lepton_selection/lepton_selection_dm1p92/none_Muons_rlj_endcape.pdf}  \\
\caption[Spatial seperation between reconstructed muons and the leading jet $\DR(\jmath_1,\mu)$]{Spatial seperation between reconstructed muons with loose ID and the leading jet $\DR(\jmath_1,\mu)$ for $\dm=5.63\GeV$ (top) and $\dm=1.92\GeV$ (bottom) in the inclusive case (left), barrel (middle) and endcaps (right).}
\label{fig:muons-dr-lj}
\end{figure}

Since the muons in the endcaps have lower \pt than the muons in the barrel, which is only able to reconstruct muons with $\pt>3\GeV$, the purity in the endcaps is much lower than the purity in the barrel, and the selection we are constructing here attempts to purify the muons further. Just as in the electrons case, we select muons with $\DR(\jmath_1,\mu)>0.4$, and that selection will apply for the rest of the section.

Next we turn into the \gls{pt} distributions. We apply the previous cut of $\DR(\jmath_1,\mu)>0.4$. As we've already seen in~\ref{sec:muon-eta-pt}, the \pt distribution depends strongly on \gls{dm}, and we try to favor the low \gls{dm} acceptance in order to be more sensitive to it. The \pt distributions we see in~\ref{fig:muons-selectrion-pt} suggest a cut identical to the electron case of $\pt<15\GeV$. It is worth mentioning that the \pt of the muons are fed into the training of the \gls{bdt} for further refinement, and therefore the exact value is being determined here quite loosely. The actual maximum value of the \pt of the muons will depend on the \gls{bdt} cut being used to define the signal region.

\begin{figure}[!htb]
\centering
\includegraphics[width=0.32\linewidth]{plots/lepton_selection/lepton_selection_dm5p63/none_Muons_pt.pdf} \,
\includegraphics[width=0.32\linewidth]{plots/lepton_selection/lepton_selection_dm5p63/none_Muons_pt_barrel.pdf}
\includegraphics[width=0.32\linewidth]{plots/lepton_selection/lepton_selection_dm5p63/none_Muons_pt_endcape.pdf}  \\
\includegraphics[width=0.32\linewidth]{plots/lepton_selection/lepton_selection_dm1p92/none_Muons_pt.pdf} \,
\includegraphics[width=0.32\linewidth]{plots/lepton_selection/lepton_selection_dm1p92/none_Muons_pt_barrel.pdf}
\includegraphics[width=0.32\linewidth]{plots/lepton_selection/lepton_selection_dm1p92/none_Muons_pt_endcape.pdf}  \\
\caption[Reconstructed muons \pt]{Reconstructed muons \pt distibution with loose ID for $\dm=5.63\GeV$ (top) and $\dm=1.92\GeV$ (bottom) in the inclusive case (left), barrel (middle) and endcaps (right). Cuts of $\DR(\jmath_1,\mu)>0.4$ and $\pt<15\GeV$ are applied.}
\label{fig:muons-selectrion-pt}
\end{figure}

We can see the feature discussed earlier, whereby the endcaps being able to reconstruct muons with lower \pt, and therefore has worse purity than the barrel, being reiterated here. It must be stressed that worse purity is due to a much higher efficiency, and therefore, as long as we can purify it further, is not necessarily a bad thing. We see however, that the bulk of the non-signal muons populate the region of $\pt<2\GeV$, and the ratio of signal muons to non-signal muons is very low in that region. We therefore make an additional cut of $\pt>2\GeV$. Another way of looking at the effect of this cut is by looking at the $\abs{\eta}$ distribution before and after the \pt cut, which can be seen in~\ref{fig:muons-selection-eta}.

\begin{figure}[!htb]
\centering
\includegraphics[width=0.48\linewidth]{plots/lepton_selection/lepton_selection_dm5p63/none_Muons_Eta.pdf} \,
\includegraphics[width=0.48\linewidth]{plots/lepton_selection/lepton_selection_dm5p63/none_Muons_Eta_after_pt.pdf} \\
\includegraphics[width=0.48\linewidth]{plots/lepton_selection/lepton_selection_dm1p92/none_Muons_Eta.pdf}  \,
\includegraphics[width=0.48\linewidth]{plots/lepton_selection/lepton_selection_dm1p92/none_Muons_Eta_after_pt.pdf} \\
\caption[\abs{\eta} distribution of reconstructed muons with loose ID before and after $\pt>2\GeV$ cut]{ \abs{\eta} distribution of reconstructed muons with loose ID for $\dm=5.63\GeV$ (top) and $\dm=1.92\GeV$ (bottom) without (left) and with (right) $\pt>2\GeV$ cut. Cut of $\DR(\jmath_1,\mu)>0.4$ is also applied.}
\label{fig:muons-selection-eta}
\end{figure}

We would like to see if requiring a tighter working point for the muon-identification is beneficial. The working point used in the previous distributions is loose. We turn now to check the effects of requiring either a medium working point, or a tight one. We plot two bins labeled \emph{fail} and \emph{pass}, which correspond to whether the muon passes or fails the identification criteria of a medium or tight working points.

\begin{figure}[!htb]
\centering
\includegraphics[width=0.32\linewidth]{plots/lepton_selection/lepton_selection_dm5p63/none_Muons_pt_medium.pdf} \,
\includegraphics[width=0.32\linewidth]{plots/lepton_selection/lepton_selection_dm5p63/none_Muons_pt_barrel_medium.pdf} \,
\includegraphics[width=0.32\linewidth]{plots/lepton_selection/lepton_selection_dm5p63/none_Muons_pt_endcape_medium.pdf}   \\
\includegraphics[width=0.32\linewidth]{plots/lepton_selection/lepton_selection_dm1p92/none_Muons_pt_medium.pdf} \,
\includegraphics[width=0.32\linewidth]{plots/lepton_selection/lepton_selection_dm1p92/none_Muons_pt_barrel_medium.pdf}  \,
\includegraphics[width=0.32\linewidth]{plots/lepton_selection/lepton_selection_dm1p92/none_Muons_pt_endcape_medium.pdf} \\
\caption[medium ID working point distribution of reconstructed muons]{Medium ID working point distributions of reconstructed muons for $\dm=5.63\GeV$ (top) and $\dm=1.92\GeV$ (bottom) in the inclusive \pt case (left), barrel (middle) and endcaps (right). Cuts of $\DR(\jmath_1,\mu)>0.4$, $\pt>2\GeV$ and $\pt<15\GeV$ are applied.}
\label{fig:muons-selection-id-medium}
\end{figure}

When we compare the medium working point in~\ref{fig:muons-selection-id-medium} to the tight working point in~\ref{fig:muons-selection-id-tight} we can see that the medium working point purifies the muons quite a lot and is very beneficial. However, when we look at the tight working point, we observe that we lose quite a lot of our wanted signal-muons without a significant gain in purity. We therefore choose to use the medium ID working point. 

\begin{figure}[!htb]
\centering
\includegraphics[width=0.32\linewidth]{plots/lepton_selection/lepton_selection_dm5p63/none_Muons_tight.pdf} \,
\includegraphics[width=0.32\linewidth]{plots/lepton_selection/lepton_selection_dm5p63/none_Muons_barrel_tight.pdf} \,
\includegraphics[width=0.32\linewidth]{plots/lepton_selection/lepton_selection_dm5p63/none_Muons_endcape_tight.pdf}   \\
\includegraphics[width=0.32\linewidth]{plots/lepton_selection/lepton_selection_dm1p92/none_Muons_tight.pdf} \,
\includegraphics[width=0.32\linewidth]{plots/lepton_selection/lepton_selection_dm1p92/none_Muons_barrel_tight.pdf}  \,
\includegraphics[width=0.32\linewidth]{plots/lepton_selection/lepton_selection_dm1p92/none_Muons_endcape_tight.pdf} \\
\caption[tight ID working point distribution of reconstructed muons]{Tight ID working point distributions of reconstructed muons for $\dm=5.63\GeV$ (top) and $\dm=1.92\GeV$ (bottom) in the inclusive \pt case (left), barrel (middle) and endcaps (right). Cuts of $\DR(\jmath_1,\mu)>0.4$, $\pt>2\GeV$ and $\pt<15\GeV$ are applied.}
\label{fig:muons-selection-id-tight}
\end{figure}

Our custom jet-isolation, which is described fully in~\ref{sec:isolation}, was devised mainly to reject \gls{sm} background while retaining signal. In the electron case, we have seen in~\ref{fig:electrons-selection-isolation} that it did a great job in also purifying the electron selection and replaced the need of requiring a tighter identification working point. In the case of the muons, we do rely on the a medium working point to perform this task, but we would like to also see the effects of the isolation on our signal muons. We see in~\ref{fig:muons-selection-isolation} that we pay a small price by requiring the isolation, but as will be seen in~\ref{sec:isolation}, we increase the sensitivity by rejecting a lot of \gls{sm} background in the process.

\begin{figure}[!htb]
\centering
\includegraphics[width=0.32\linewidth]{plots/lepton_selection/lepton_selection_dm5p63/none_Muons_pt_jet_iso.pdf} \,
\includegraphics[width=0.32\linewidth]{plots/lepton_selection/lepton_selection_dm5p63/none_Muons_pt_barrel_jet_iso.pdf} \,
\includegraphics[width=0.32\linewidth]{plots/lepton_selection/lepton_selection_dm5p63/none_Muons_pt_endcape_jet_iso.pdf}   \\
\includegraphics[width=0.32\linewidth]{plots/lepton_selection/lepton_selection_dm1p92/none_Muons_pt_jet_iso.pdf} \,
\includegraphics[width=0.32\linewidth]{plots/lepton_selection/lepton_selection_dm1p92/none_Muons_pt_barrel_jet_iso.pdf}  \,
\includegraphics[width=0.32\linewidth]{plots/lepton_selection/lepton_selection_dm1p92/none_Muons_pt_endcape_jet_iso.pdf} \\
\caption[jet-isolation distribution of reconstructed muons]{Jet-isolation distributions of reconstructed muons with medium ID for $\dm=5.63\GeV$ (top) and $\dm=1.92\GeV$ (bottom) in the inclusive \pt case (left), barrel (middle) and endcaps (right). Cuts of $\DR(\jmath_1,\mu)>0.4$, $\pt>2\GeV$ and $\pt<15\GeV$ are applied.}
\label{fig:muons-selection-isolation}
\end{figure}

We summaries this section with the full selection of the analysis muons:
\begin{itemize}
\item $2<\pt<15\GeV$
\item $\abs{\eta} < 2.4$
\item $\DR(\jmath_1,\mu)>0.4$
\item medium ID working point
\item pass jet-isolation
\end{itemize}

\clearpage

\subsection{Scale factors}

In~\ref{sec:object-selection-electrons} and~\ref{sec:muon-selection} we have studied the selection we apply to electrons and muons respectively. We have therewith made a choice about identification working point. We have used simulation exclusively to draw conclusions about the identification efficiency of the leptons. If we rely on \gls{mc} simulation,  that will produce large systematic errors due to imperfections in modeling both the data and the detector response. We therefore wish to measure the identification efficiency in data, in order to correct the simulation's potentially false efficiency rate. \emph{Efficiency} is defined as how probable it is to reconstruct or identify a lepton. For a lepton $\ell$ the identification efficiency is defined as:

\begin{equation}
\varepsilon_{\ell}^{\mathrm{ID}} = \frac{N_{\ell}(\mathrm{ID})}{N_{\ell}(\mathrm{produced})}
\end{equation}

In \gls{mc} simulation, the number of leptons produced is the same as the number of leptons generated. In data, one must measure it using a data-driven method. Once the efficiencies have been measured both in simulation and in data, a correction factor named \gls{sf} can be applied to the simulation in order to correct for any discrepancies that might arise. The scale factors are defined as the ratio between the efficiency in data to the efficiency in simulation:

\begin{equation}
\mathrm{SF}_{\ell}^{\mathrm{ID}}=\frac{\varepsilon_{\ell}^{\mathrm{ID,Data}}}{\varepsilon_{\ell}^{\mathrm{ID,MC}}},
\end{equation}

dropping the superscript ID we get:

\begin{equation}
\mathrm{SF}_{\ell}=\frac{\varepsilon_{\ell}^{\mathrm{Data}}}{\varepsilon_{\ell}^{\mathrm{MC}}}.
\end{equation}

Once the relevant \gls{sf} have been determined, they are applied for every lepton passing the object selection in the event. The scale factors for loose-ID electrons in our \pt range have been measured centrally by the relevant working group and are applied to the selected electrons. As was determined in~\ref{sec:muon-selection}, our analysis signal muon's lower \pt threshold is $2\GeV$ which is low. Scale factors for medium ID leptons with $\pt\geq 2\GeV$ were computed centrally by the Muon \gls{pog}. The scale factors, however, while matching our muons' \pt range and identification working point, were computed by requiring $\DR > 0.5$  between the muons~\cite{muon-id-sf-2016,muon-id-sf-2016-pres}. As we have seen in~\ref{sec:lepton-dr}, one of are drivers of the sensitivity is the region of $\DR < 0.5$. We would therefore like to validate the scale factors in that region. We would like to show that the efficiencies have no $\DR$ dependence, and in order to do so, we calculate the efficiencies in different $\DR$ regions.

In order to measure such efficiencies in data, one must identify desired leptons with low and easily reducible fakes. A widely used method to perform such a data-driven task is the Tag \& Probe method. In the Tag \& Probe method we examine a mass resonance such as \PZ, \JPsi or \PGU to select particles of the desired type, and probe the efficiency of a particular selection criterion on those particles. The mass resonance will then decay into two same-flavor opposite charged pair of leptons and will form a peak on top of a background. Since we are interested in measuring the efficiency of low \pt muons, we choose to look at dimuon events around the \JPsi mass window. In a dimuon event, we describe one muon as a `tag' and the other as a `probe'. The tag muon is selected with a very tight selection which results in very high certainty that the object corresponds to a real muon produced. The probe is given a loose selection, but since it is constrained to be consistent with a product of a \JPsi, it is almost certain that it originates from a real muon too. Since the shape of the \JPsi is a peak over a background, the background is easily removed by a fit. The probe is then subjected to cuts or constraints which are used to measure a particular efficiency. As stated, in this study we want to show that the efficiency to identify a muon with medium ID working point from a track has no $\DR$ depandance. Therefore, the efficiency we are looking for is defined as:

\begin{equation}
\varepsilon_{\mu}^{\mathrm{ID}} = \frac{N_{\mu}^\mathrm{ID}}{N_{t}}.
\end{equation}

The probe we are using in the denominator is a track passing a loose selection, while the probe track in the numerator is required to match a medium ID working point muon. The number of objects passing a selection is determined by a fit to data and \gls{mc}, to measure the corresponding efficiencies in data and \gls{mc}.

This study is done for year 2016. For the \gls{mc}, we are using the 2016 samples listed in~\ref{sec:sm-mc}. For data, we are  using a single electron trigger, in order for the tagged muon to be independent from the triggered object. The data set is measured to correspond to 36.02\fbinv using the BRIL Work Suite~\cite{bril}. The following trigger paths are used:

\begin{itemize}
\item \texttt{HLT\_Ele27\_WPTight\_Gsf\_v*},
\item \texttt{HLT\_Ele27\_eta2p1\_WPLoose\_Gsf\_v*},
\item \texttt{HLT\_Ele32\_WPTight\_Gsf\_v*},
\item \texttt{HLT\_Ele35\_WPTight\_Gsf\_v*}.
\end{itemize}

We then select an offline loose ID electron with $\pt>27\GeV$. The requirements to select a tag \& probe pair are defined in table~\ref{tab:tag-probe-def}.

\begin{table}[!htb]
	\centering
	\label{tab:tag-probe-def}
		\caption{Selection criteria for Tags and Probes}
		%\vspace{1mm}
			\begin{tabular}{l|l} \hline
			Tag & Probe \\ \hline
			medium ID muon & isolated track\\
			$\pt \geq 5\GeV$ & $2\leq\pt\leq 20\GeV$  ($ \pt\geq 3 \GeV $ for barrel) \\
			$\abs{\eta}<2.4$ & opposite-sign in invariant mass window $[2.5,3.5]\GeV$ \\ \hline
			\end{tabular}
\end{table}

A fit is then performed in an invariant mass window around the \JPsi window of $[2.5,3.5]\GeV$. The signal fit is using a crystal ball function and the continuum is fit with a 6th order polynomial. The fit is repeated twice, where the denominator is done with probe tracks, and the numerator is using medium ID muons that have been matched to said tracks. The \DR~range has been split into 3, and $\abs{\eta}$ of the muons has been split into barrel ($\abs{\eta}<1.2$) and endcaps ($1.2<\abs{\eta}<2.4$). Simulation fits are shown in~\ref{fig:tb-barrel-simulation} for barrel, and~\ref{fig:tb-endcaps-simulation} for endcaps. Data fits are shown in~\ref{fig:tb-barrel-data} for barrel, and~\ref{fig:tb-endcaps-data} for endcaps.

\begin{figure}[!htbp]
\centering
\includegraphics[width=0.32\linewidth]{plots/jpsi_muons_fit_bg_delta_r_single_electron/none_invMass_0_0.3_0_1.2.pdf} \,
\includegraphics[width=0.32\linewidth]{plots/jpsi_muons_fit_bg_delta_r_single_electron/none_invMass_0.3_0.5_0_1.2.pdf}  \,
\includegraphics[width=0.32\linewidth]{plots/jpsi_muons_fit_bg_delta_r_single_electron/none_invMass_0.5_1.5_0_1.2.pdf} \\
\includegraphics[width=0.32\linewidth]{plots/jpsi_muons_fit_bg_delta_r_single_electron/none_id_invMass_0_0.3_0_1.2.pdf} \,
\includegraphics[width=0.32\linewidth]{plots/jpsi_muons_fit_bg_delta_r_single_electron/none_id_invMass_0.3_0.5_0_1.2.pdf}  \,
\includegraphics[width=0.32\linewidth]{plots/jpsi_muons_fit_bg_delta_r_single_electron/none_id_invMass_0.5_1.5_0_1.2.pdf} \\
\caption[Simluation barrel muons fits]{Simluation barrel muons fits for denominator (top) and numerator (bottom) for $0<\DR<0.3$  (left), $0.3<\DR<0.5$ (center), $0.5<\DR<1.5$ (right)}
\label{fig:tb-barrel-simulation}
\end{figure}

\begin{figure}[!htbp]
\centering
\includegraphics[width=0.32\linewidth]{plots/jpsi_muons_fit_bg_delta_r_single_electron/none_invMass_0_0.3_1.2_2.4.pdf} \,
\includegraphics[width=0.32\linewidth]{plots/jpsi_muons_fit_bg_delta_r_single_electron/none_invMass_0.3_0.5_1.2_2.4.pdf} \,
\includegraphics[width=0.32\linewidth]{plots/jpsi_muons_fit_bg_delta_r_single_electron/none_invMass_0.5_1.5_1.2_2.4.pdf} \\
\includegraphics[width=0.32\linewidth]{plots/jpsi_muons_fit_bg_delta_r_single_electron/none_id_invMass_0_0.3_1.2_2.4.pdf} \,
\includegraphics[width=0.32\linewidth]{plots/jpsi_muons_fit_bg_delta_r_single_electron/none_id_invMass_0.3_0.5_1.2_2.4.pdf} \,
\includegraphics[width=0.32\linewidth]{plots/jpsi_muons_fit_bg_delta_r_single_electron/none_id_invMass_0.5_1.5_1.2_2.4.pdf}  \\
\caption[Simluation endcaps muons fits]{Simluation endcaps muons fits for denominator (top) and numerator (bottom) for $0<\DR<0.3$  (left), $0.3<\DR<0.5$ (center), $0.5<\DR<1.5$ (right)}
\label{fig:tb-endcaps-simulation}
\end{figure}

\begin{figure}[!htbp]
\centering
\includegraphics[width=0.32\linewidth]{plots/jpsi_muons_fit_data_delta_r_single_electron/none_invMass_0_0.3_0_1.2.pdf} \,
\includegraphics[width=0.32\linewidth]{plots/jpsi_muons_fit_data_delta_r_single_electron/none_invMass_0.3_0.5_0_1.2.pdf}  \,
\includegraphics[width=0.32\linewidth]{plots/jpsi_muons_fit_data_delta_r_single_electron/none_invMass_0.5_1.5_0_1.2.pdf} \\
\includegraphics[width=0.32\linewidth]{plots/jpsi_muons_fit_data_delta_r_single_electron/none_id_invMass_0_0.3_0_1.2.pdf} \,
\includegraphics[width=0.32\linewidth]{plots/jpsi_muons_fit_data_delta_r_single_electron/none_id_invMass_0.3_0.5_0_1.2.pdf}  \,
\includegraphics[width=0.32\linewidth]{plots/jpsi_muons_fit_data_delta_r_single_electron/none_id_invMass_0.5_1.5_0_1.2.pdf} \\
\caption[Data barrel muons fits]{Data barrel muons fits for denominator (top) and numerator (bottom) for $0<\DR<0.3$  (left), $0.3<\DR<0.5$ (center), $0.5<\DR<1.5$ (right)}
\label{fig:tb-barrel-data}
\end{figure}

\begin{figure}[!htbp]
\centering
\includegraphics[width=0.32\linewidth]{plots/jpsi_muons_fit_data_delta_r_single_electron/none_invMass_0_0.3_1.2_2.4.pdf} \,
\includegraphics[width=0.32\linewidth]{plots/jpsi_muons_fit_data_delta_r_single_electron/none_invMass_0.3_0.5_1.2_2.4.pdf} \,
\includegraphics[width=0.32\linewidth]{plots/jpsi_muons_fit_data_delta_r_single_electron/none_invMass_0.5_1.5_1.2_2.4.pdf} \\
\includegraphics[width=0.32\linewidth]{plots/jpsi_muons_fit_data_delta_r_single_electron/none_id_invMass_0_0.3_1.2_2.4.pdf} \,
\includegraphics[width=0.32\linewidth]{plots/jpsi_muons_fit_data_delta_r_single_electron/none_id_invMass_0.3_0.5_1.2_2.4.pdf} \,
\includegraphics[width=0.32\linewidth]{plots/jpsi_muons_fit_data_delta_r_single_electron/none_id_invMass_0.5_1.5_1.2_2.4.pdf}  \\
\caption[Data endcaps muons fits]{Data endcaps muons fits for denominator (top) and numerator (bottom) for $0<\DR<0.3$  (left), $0.3<\DR<0.5$ (center), $0.5<\DR<1.5$ (right)}
\label{fig:tb-endcaps-data}
\end{figure}

The efficiencies and corresponing scale factors can be seen in ~\ref{fig:tb-eff-sf}. The scale factors are statistically consistent with unity, and show no discernible $\DR$ dependence. A similar study has been carried out with simulation and data for 2017 and 2018 in~\cite{muon-id-sf-2017-8} and did not observe a $\DR$ dependence either. The recommendation from the \gls{pog} as a results of these studies are to use the calculated scale factors provided by them with an additional systematic of 1\% for muons with $\pt<20\GeV$.

\begin{figure}[!htbp]
\centering
\includegraphics[width=0.48\linewidth]{plots/scale_factors/barrelDeltaRSingleElectron.pdf} \,
\includegraphics[width=0.48\linewidth]{plots/scale_factors/endcapsDeltaRSingleElectron.pdf}  \\
\includegraphics[width=0.48\linewidth]{plots/scale_factors/barrelDeltaRisoScaleFactorsSingleElectron.pdf} \,
\includegraphics[width=0.48\linewidth]{plots/scale_factors/endcapsDeltaRisoScaleFactorsSingleElectron.pdf} \\
\caption[Efficiencies and scale factors]{Efficiencies (top) and scale factors (bottom) for barrel muons (left) and endcaps muons (right).}
\label{fig:tb-eff-sf}
\end{figure}

\clearpage

\subsection{Missing transverse energy}
\label{subsec:met}

The importance of measuring the missing transverse momentum (or energy) in this analysis has been discussed in~\ref{subsec:signal-met-mht}. We concluded that the missing transverse momentum is an essential ingredient in our analysis strategy. It is used in order to trigger events online, as well as in our offline event level selection in order to boost sensitivity. Two standard measured of the momentum imbalance in the events are \VEtmiss (or equivalently by a different symbol \ptvecmiss) and \htvecmiss. Although \VEtmiss and \ptvecmiss have different symbols, and referred to by different names (missing transverse energy and missing transverse momentum respectively), they are defined in the same way and are used interchangeably. Mathematically speaking, \VEtmiss is defined in the following way:

\begin{equation}
\VEtmiss = \ptvecmiss = -\sum_i \ptvec(i)
\end{equation}

where the summation is done on all particle flow candidates. Therefore, it is a measure of particles that escape detection, such as weakly interacting particles. The missing transverse energy is highly sensitive to any mismeasurements of the visible
particles, as well as additional energy deposits from \gls{pu}, detector noise etc. Therefore, this observable is further corrected to mitigate \gls{pu} effects, as well as jet energy response. For those corrections, jets with \pt greater than $10\GeV$ are being considered. Full details of the corrections are given in~\cite{met_performance}.

An alternative measurement to the missing transverse momentum is \htvecmiss, sometimes referred to as \emph{missing hardronic activity}. Rather than looking at all particle flow candidates in the sum, it takes into account only jets with \pt greater than $30\GeV$ with $\abs{\eta}<5$:

\begin{equation}
\htvecmiss = -\sum_i^{\mathrm{jets}} \ptvec(i)
\end{equation}

In this analysis we favor using the observable \htvecmiss over \VEtmiss, since, as we see in our definition of the jet isolation in~\ref{sec:isolation}, we use jets with \pt greater than $30\GeV$, and we keep the range of $\pt\in [15,30]\GeV$ of jets to define a side band which we then use for our data-driven background estimation method in~\ref{sec:jetty-background-estimation}.  For both observables, \VEtmiss and \htvecmiss, we can define the equivalent scalar quantities, \MET and \mht respectively, by taking the magnitude of their vectorial counterpart.

\subsection{Jets}
\label{subsec:jets}

Jets reconstruction and identification is described in~\ref{sec:reco-jets}. Jets used in the analysis are reconstructed based on a clustering of the \gls{pf} candidates using $\FASTJET$ with the anti-$\kt$ algorithm~\cite{Cacciari_2008_antikt} with the size parameter 0.4. Tagging of $b$ quark jets is done using the so-called \gls{csv} algorithm based on a multivari­ate technique \DEEPCSV with a medium working point. Each jet is required to have $\pt>30\GeV$ and $\abs{\eta}<2.4$.

\clearpage
\subsection{Tracks and multivariate selection }
\label{sec:track-bdt}
The leptons \ellell produced in our decay \neuttdecay tend to have mostly very low transverse momentum \pt. We have seen in~\ref{sec:reconstruction-and-identification} that the identification and reconstruction of the muons get worst with lower \pt. Therefore, the aim of the exclusive track category is to try and regain lost leptons that didn't make the reconstruction or identification process. Since the tracking efficiency at our \pt ranges is well above 99\%, as we've seen in~\ref{sec:track-reconstruction}, we can try and pick up the track that corresponds to our missing lepton by applying our knowledge about the event kinematics specific to our signal. We are therefore, not enhancing the lepton identification in the general case, but identifying missed tracks in our specific signal signature.

Since each event contains many tracks, we need a method to identify in the signal event, which track corresponds to our miss-identified lepton. In order to achieve this goal we train a \glsreset{bdt}\gls{bdt}. We train 4 \glspl{bdt} which correspond to the lepton flavor, Muons or Electrons, and for each, we train separately for phase 0 (2016) and phase 1 (2017-2018) of the tracker. All \glspl{bdt} use the same structure of 200 trees with a maximum depth of 3, with the TMVA package~\cite{tmva}. The \gls{bdt} training is performed with AdaBoost and GiniIndex separation. We are taking all other values as the defaults set by the TMVA package.

For the training we take tracks from a pool of our privately produced \FASTSIM signal simulations which were listed in~\ref{sec:signal-simulation}. Of those, we are selecting the full range of simulated higgsino parameter $\mu$ (or the mass of \PSGcpmDo in case of phase 1), but only the range of \dm we want to be most sensitive to. In phase 0, we select $\dmo\in [0.3,4.3]\GeV$ and $\mu\in [100-130]\GeV$. In phase 1 we select $\dmpm \in [0.3-4.6]\GeV$ and $\mu\in [100-500]\GeV$. Those signal events we then split into signal tracks, \ie, tracks originating from the decay \neuttdecay and been matched to the missing generated lepton in the generator level particles collection, and background tracks which do not match our wanted leptons. Therefore, our \glspl{bdt} are useful to reject in-signal background of unwanted tracks. The samples for muons contain 9408 (10964) signal tracks and 99996 (151380) background tracks for phase 0 (phase 1). For electrons the samples contain 2364 (2288) signal tracks and 104065 (159713) background tracks for phase 0 (phase 1). The training samples are then tested against the test samples of equal size. The distributions of the testing samples overlay on the training samples are seen in~\ref{fig:track-bdt-output}.

A pre-selection is applied to all tracks in the collection obtained by the standard track reconstruction sequences. The pre-selection ensures a set of properly-reconstructed, isolated and prompt tracks, whose trajectories pass through the region nearby the primary vertex with the largest sum of charged-tracks, jets and missing energy values (PV): 

\begin{itemize}
\item $ \pt > 1.9 \GeV$
\item $ \abs{\eta} < 2.4$
\item track $\text{iso}_\text{rel}  < 0.1$, using \DR(track, other tracks) $< 0.3$
\item $d_{xy}(\text{track, PV}) < 0.02\, \cm$ w.r.t the PV
\item $d_z(\text{track, PV}) < 0.02\, \cm$ w.r.t the PV
\item no match to an electron or muon within a cone of size 0.01
\end{itemize}

For the training we use 10 variables listed in decreasing order of their ranking (in the muon case of phase 0) listen in~\ref{tab:track-bdt-variables}.

\begin{table}[!htb]
	\centering
	\label{tab:track-bdt-variables}
		\caption{Track BDT input variables}
		%\vspace{1mm}
			\begin{tabular}{cll} \hline
			Rank & Variable & Description \\ \hline
			1 & $\DR\left(t, \ell\right)$ & $t$ is the track and $\ell$ the lepton\\
			2 & $\abs{\Delta \eta \left(t, \ell\right) }$ & \\
			3 & $\pt(\ell)$ & \\
			
			4 & $\abs{\Delta\phi\left(t, \htvecmiss \right)}$ & \\
			5 & $\abs{\Delta\eta\left(t, j_1 \right)}$ & $j_1$ is the leading jet\\
			6 & $\abs{\Delta\phi\left(t, \ell \right)}$ & \\
			7 & $\abs{\eta\left(t\right)}$ & \\
			8 & $\abs{\eta\left(\ell\right)}$ & \\
			9 & $\DR\left(\ell, j_1\right)$ & \\
			10 & $\mlt$ & invariant mass \\ 
			\hline
			\end{tabular}
\end{table}

Distribution for the input variables showing the signal tracks in blue, and background tracks in red, are seen in~\ref{fig:muon-track-bdt-inputs}.

\begin{figure}[!htb]
\centering
\includegraphics[width=0.32\linewidth]{plots/track_bdt_inputs_muons/none_deltaRLL.pdf} \,
\includegraphics[width=0.32\linewidth]{plots/track_bdt_inputs_muons/none_deltaEtaLL.pdf} \,
\includegraphics[width=0.32\linewidth]{plots/track_bdt_inputs_muons/none_lepton.Pt__.pdf} \\

\includegraphics[width=0.32\linewidth]{plots/track_bdt_inputs_muons/none_deltaPhiMht.pdf} \,
\includegraphics[width=0.32\linewidth]{plots/track_bdt_inputs_muons/none_deltaEtaLJ.pdf} \,
\includegraphics[width=0.32\linewidth]{plots/track_bdt_inputs_muons/none_abs_deltaPhiLL_.pdf}  \\

\includegraphics[width=0.32\linewidth]{plots/track_bdt_inputs_muons/none_abs_track.Eta___.pdf} \,
\includegraphics[width=0.32\linewidth]{plots/track_bdt_inputs_muons/none_abs_lepton.Eta___.pdf} \,
\includegraphics[width=0.32\linewidth]{plots/track_bdt_inputs_muons/none_deltaRLJ.pdf}  \\


\includegraphics[width=0.32\linewidth]{plots/track_bdt_inputs_muons/none_invMass.pdf}  \\


\caption[Muon track BDT training inputs]{Distibutions of the inputs for training for the track BDT in the Muon exclusive track category.}
\label{fig:muon-track-bdt-inputs}
\end{figure}

The output of the training of the 4 \glspl{bdt} can be then seen in~\ref{fig:track-bdt-output}. The testing distributions are laid on top of the training sample of the BDT. No obvious over-training is seen. The ROC curves are then plotted in~\ref{fig:track-bdt-roc}. The red point shows the efficiency of the signal and background tracks of the BDT cut which is chosen to be 0.0. We get quite a good separation of signal tracks from fake tracks, as can be seen by the relatively high signal efficiency of over 90\% (86\%) for muons (electrons) and background rejection of around 86\% (76\%) for muons (electrons).

\begin{figure}[!htb]
\centering
\includegraphics[width=0.48\linewidth]{plots/track_bdt/overtraining_Tracks_Muons_Phase_0.pdf} \,
\includegraphics[width=0.48\linewidth]{plots/track_bdt/overtraining_Tracks_Muons_Phase_1.pdf}  \\
\includegraphics[width=0.48\linewidth]{plots/track_bdt/overtraining_Tracks_Electrons_Phase_0.pdf} \,
\includegraphics[width=0.48\linewidth]{plots/track_bdt/overtraining_Tracks_Electrons_Phase_1.pdf} \\
\caption[Track BDT output plots]{Track BDT output plots for Muons (top) and Electrons (bottom) in phase 0 (left) and phase 1 (right). Blue shows signal tracks, while Red are fake tracks. Test sample overlay on top of training sample.}
\label{fig:track-bdt-output}
\end{figure}

\begin{figure}[!htb]
\centering
\includegraphics[width=0.48\linewidth]{plots/track_bdt/roc_Tracks_Muons_Phase_0.pdf} \,
\includegraphics[width=0.48\linewidth]{plots/track_bdt/roc_Tracks_Muons_Phase_1.pdf}  \\
\includegraphics[width=0.48\linewidth]{plots/track_bdt/roc_Tracks_Electrons_Phase_0.pdf} \,
\includegraphics[width=0.48\linewidth]{plots/track_bdt/roc_Tracks_Electrons_Phase_1.pdf} \\
\caption[Track BDT ROC curves]{Track BDT ROC curves for Muons (top) and Electrons (bottom) in phase 0 (left) and phase 1 (right). Minimum working point showed as a red dot.}
\label{fig:track-bdt-roc}
\end{figure}

After the training process, we select the track with the maximum \gls{bdt} score as our missing lepton in the event. We consider only events with a track with a score of greater then 0.0, which corresponds to the red dot in the ROC curves in~\ref{fig:track-bdt-roc}.

\clearpage
\subsection{Isolation}
\label{sec:isolation}

The leptons that are produced from the neutralino decay \neuttdecay are typically clean and isolated with very little hardronic activity around them. That is due to the fact that the only jets in the event come from initial state radiation which provides a boost to the produced electoweakinos in the other direction. Therefore, the leptons originating from those electoweakinos will not exist in proximity to said jets. We can exploit this signature in order to discriminate between our signal and background originating from standard model processes in association with jets. At \gls{cms} various standard isolation criteria were developed and used. The three most widely used isolation criteria are track isolation~\cite{muon-pog-recommendations}, relative isolation ($\mathrm{RelIso}$), which was first described in~\cite{Chatrchyan_2011}, and a modified version refereed to as relative mini-isolation ($\mathrm{miniRelIso}$) described in~\cite{Rehermann_2011}.

Track isolation is defined as the \pt sum of all tracks around a given track (or lepton) in a fixed cone size of 0.3:

\begin{equation}
\text{Track Isolation}_\ell = \frac{ \sum_{\substack{\text{tracks from PV}\\ \text{in } \DR<0.3}}\pt}{ \pt(\ell) } 
\end{equation}

Since we are summing tracks, we take into account only charged particles. Another widely used isolation is the relative isolation which uses a cone size of 0.4 and defined as:

\begin{equation}
\text{RelIso}_\ell = \frac{ \sum_{\substack{\text{charged}\\\text{hadrons}\\\text{from PV}}}\pt + \max\left(0, \sum_{\substack{\text{neutral}\\\text{hadrons}}}\et + \sum_{\text{photons}}\et - 0.5\cdot \sum_{\substack{\text{charged}\\\text{hadrons}\\\text{from PU}}}\pt \right) }{ \pt(\ell) }
\end{equation}

The last term in the definition is correction due to \gls{pu} effects. A lepton is said to pass isolation if it is required to have a small RelIso value. The third widely used isolation criterion is the mini-realtive isolation miniRelIso, and is a modified version of the relative isolation. The difference is that in the mini-isolation, the cone size depends on the \pt of the lepton in the following way:

\begin{equation}
R = 
\begin{dcases}
0.2 & \pt(\ell)\leq 50\GeV \\
\frac{10\GeV}{\pt(\ell)} & \pt(\ell) \in (50\GeV , 200\GeV) \\
0.05 & \pt(\ell)\geq 200\GeV
\end{dcases}
\end{equation}

The variable size cone allows to recover efficiency when leptons are produced in the decay chain of a boosted object. In such cases, it is likely that, when the boost is large, the lepton overlaps with a jet produced in the same decay chain, failing a standard isolation cut. 

The drawback of the standard isolation criteria is that if two leptons are in proximity of each other, as is often the case in our signal, they will spoil each other isolation. We have seen in~\ref{sec:lepton-dr} that in order to gain access to low \dm model-points, we need to also gain access to the $\DR<0.3$ phase-space region. That means that requiring any of the standard isolation criterion will result in rejecting valuable signal events. We therefore propose an alternative isolation criterion that will help retain some of the wanted phase-space while rejecting the majority of the standard model background. Another crucial use for an alternative isolation criterion is background estimation, as is described in~\ref{sec:jetty-background-estimation} Here we describe the steps to construct the alternative \emph{jet isolation}. It is defined for each lepton flavor individually, since the leptons produced are of same flavor. The steps are as follows (introducing parameters $p$ and $r$):

\begin{enumerate}
\item Create a copy set of the jets collection
\item Create a \emph{lepton-corrected} set of the jets by subtracting all leptons within the jet cone of 0.4
\item Clean the set of jets by keeping only jets with positive energy and \pt greater than threshold $p$
\item Lepton is said to pass \emph{jet-isolation} if it does not lie within a cone of size $r$ from a \emph{lepton-corrected} jet
\item Lepton is said to fail \emph{jet-isolation} for background estimation if it fails \emph{jet-isolation}, and the original jet closest to it has $15<\pt<30\GeV$ (see~\ref{sec:jetty-background-estimation} for use of such lepton)
\end{enumerate}

The main idea behind the definition of the jet-isolation is that we want to reject leptons with hadronic activity around them, but not losing a lepton that is close to another same-flavor lepton. The process we described introduced two free parameters, namely, the \pt threshold of the lepton-corrected jets responsible for failing a lepton's isolation $p$, and the cone size $r$ which determines how close a corrected jet is allowed to be to a lepton. In order to choose the thresholds for these parameters, a scan has been performed ranging $p\in [0,20]\GeV$ and $r\in[0.4,0.6]$. The scan is used to compare different criteria for the optimal values. Since we are also using the isolation for our background estimation, some of the criteria we are interested in stem from that method. The criteria we are interested in are signal efficiency (which we want to be high), background efficiency (which we want to be low), signal contamination in our control-regions (ideally low), jetty-background transfer factor (ideally less than 1), and lastly, the significance, which is computed taking into account transfer factor error on the background (which ideally we want to maximize). To demonstrate the effect these parameter space has on these criteria, we can look at a scan done for the muons using 2016 \gls{mc} and data.

\begin{table}[!htb]
	\centering
	\label{tab:iso-scan-signal-efficiency}
		\caption{Signal Efficiency}
		%\vspace{1mm}
			\begin{tabular}{cc|ccccc}
    			&\multicolumn{1}{c}{} & \multicolumn{5}{c}{$\Delta R$} \\
    && 0.4 & 0.45 & 0.5 & 0.55 & 0.6 \\
    \cline{2-7}
    & 0 & 0.38 & 0.37 & 0.36 & 0.35 & 0.35 \\
& 1 & 0.39 & 0.38 & 0.37 & 0.37 & 0.36 \\
& 5 & 0.65 & 0.64 & 0.63 & 0.62 & 0.60 \\
& 6 & 0.71 & 0.70 & 0.69 & 0.67 & 0.66 \\
& 7 & 0.77 & 0.76 & 0.74 & 0.73 & 0.72 \\
& 8 & 0.82 & 0.82 & 0.80 & 0.78 & 0.77 \\
\smash{\rotatebox[origin=c]{90}{\pt}} & 9 & 0.87 & 0.86 & 0.85 & 0.84 & 0.82 \\
& 10 & 0.89 & 0.89 & 0.87 & 0.86 & 0.85 \\
& 10.5 & 0.90 & 0.90 & 0.89 & 0.88 & 0.87 \\
& 11 & 0.92 & 0.92 & 0.91 & 0.90 & 0.89 \\
& 11.5 & 0.93 & 0.92 & 0.91 & 0.91 & 0.90 \\
& 12 & 0.94 & 0.93 & 0.92 & 0.91 & 0.90 \\
& 12.5 & 0.94 & 0.94 & 0.93 & 0.92 & 0.91 \\
& 13 & 0.95 & 0.95 & 0.94 & 0.93 & 0.93 \\
& 15 & 0.98 & 0.98 & 0.97 & 0.97 & 0.97 \\
& 20 & 1.00 & 1.00 & 1.00 & 0.99 & 0.99
  \end{tabular}
\end{table}

\begin{table}[!htb]
	\centering
	\label{tab:iso-scan-bg-efficiency}
		\caption{Background Efficiency}
		%\vspace{1mm}
			\begin{tabular}{cc|ccccc}
    			&\multicolumn{1}{c}{} & \multicolumn{5}{c}{$\Delta R$} \\
    && 0.4 & 0.45 & 0.5 & 0.55 & 0.6 \\
    \cline{2-7}
& 0 & 0.08 & 0.07 & 0.06 & 0.06 & 0.06 \\ 
& 1 & 0.08 & 0.07 & 0.06 & 0.06 & 0.06 \\ 
& 5 & 0.12 & 0.12 & 0.10 & 0.09 & 0.09 \\ 
& 6 & 0.15 & 0.14 & 0.12 & 0.11 & 0.11 \\ 
& 7 & 0.18 & 0.16 & 0.15 & 0.14 & 0.12 \\ 
& 8 & 0.20 & 0.18 & 0.17 & 0.17 & 0.15 \\ 
\smash{\rotatebox[origin=c]{90}{\pt}} & 9 & 0.25 & 0.23 & 0.19 & 0.18 & 0.17 \\ 
& 10 & 0.26 & 0.25 & 0.22 & 0.19 & 0.18 \\ 
& 10.5 & 0.27 & 0.24 & 0.23 & 0.20 & 0.19 \\ 
& 11 & 0.29 & 0.26 & 0.24 & 0.22 & 0.20 \\ 
& 11.5 & 0.28 & 0.27 & 0.24 & 0.23 & 0.21 \\ 
& 12 & 0.29 & 0.27 & 0.26 & 0.24 & 0.23 \\ 
& 12.5 & 0.31 & 0.28 & 0.26 & 0.26 & 0.23 \\ 
& 13 & 0.33 & 0.29 & 0.27 & 0.27 & 0.24 \\ 
& 15 & 0.36 & 0.33 & 0.30 & 0.29 & 0.26 \\ 
& 20 & 0.45 & 0.41 & 0.39 & 0.36 & 0.37 \\ 
  \end{tabular}
\end{table}

\begin{table}[!htb]
	\centering
	\label{tab:iso-scan-transfer-factor}
		\caption{Transfer Factor}
		%\vspace{1mm}
			\begin{tabular}{cc|ccccc}
    			&\multicolumn{1}{c}{} & \multicolumn{5}{c}{$\Delta R$} \\
    && 0.4 & 0.45 & 0.5 & 0.55 & 0.6 \\
    \cline{2-7}
& 0 & 0.19 & 0.16 & 0.13 & 0.13 & 0.13 \\ 
& 1 & 0.18 & 0.16 & 0.14 & 0.13 & 0.13 \\ 
& 5 & 0.31 & 0.30 & 0.26 & 0.23 & 0.22 \\ 
& 6 & 0.43 & 0.36 & 0.32 & 0.30 & 0.29 \\ 
& 7 & 0.55 & 0.48 & 0.44 & 0.40 & 0.34 \\ 
& 8 & 0.68 & 0.58 & 0.52 & 0.52 & 0.46 \\ 
\smash{\rotatebox[origin=c]{90}{\pt}} & 9 & 0.83 & 0.78 & 0.65 & 0.58 & 0.54 \\ 
& 10 & 0.99 & 0.93 & 0.76 & 0.67 & 0.62 \\ 
& 10.5 & 1.07 & 0.95 & 0.85 & 0.74 & 0.66 \\ 
& 11 & 1.19 & 1.10 & 0.93 & 0.85 & 0.73 \\ 
& 11.5 & 1.24 & 1.19 & 0.96 & 0.91 & 0.79 \\ 
& 12 & 1.34 & 1.29 & 1.09 & 0.99 & 0.91 \\ 
& 12.5 & 1.55 & 1.35 & 1.21 & 1.10 & 0.95 \\ 
& 13 & 1.70 & 1.46 & 1.27 & 1.23 & 1.09 \\ 
& 15 & 2.39 & 2.17 & 1.80 & 1.63 & 1.42 \\ 
& 20 & 6.12 & 5.86 & 4.82 & 4.13 & 3.86 \\ 
  \end{tabular}
\end{table}

\begin{table}[!htb]
	\centering
	\label{tab:iso-scan-significance}
		\caption{Significance $s/\sqrt{b+\upvarepsilon^2_b}$}
		%\vspace{1mm}
			\begin{tabular}{cc|ccccc}
    			&\multicolumn{1}{c}{} & \multicolumn{5}{c}{$\Delta R$} \\
    && 0.4 & 0.45 & 0.5 & 0.55 & 0.6 \\
    \cline{2-7}
& 0 & 4.29 & 6.08 & 6.13 & 5.89 & 5.46 \\ 
& 1 & 4.92 & 5.18 & 6.34 & 5.33 & 5.84 \\ 
& 5 & 6.44 & 5.27 & 6.20 & 8.63 & 5.98 \\ 
& 6 & 4.72 & 5.06 & 6.22 & 6.99 & 7.92 \\ 
& 7 & 4.83 & 6.55 & 5.09 & 5.63 & 6.28 \\ 
& 8 & 3.80 & 5.48 & 4.60 & 5.24 & 4.61 \\ 
\smash{\rotatebox[origin=c]{90}{\pt}} & 9 & 3.60 & 4.43 & 5.66 & 6.25 & 4.60 \\ 
& 10 & 3.37 & 4.08 & 5.57 & 4.78 & 6.23 \\ 
& 10.5 & 3.72 & 4.03 & 4.90 & 4.48 & 4.17 \\ 
& 11 & 3.05 & 3.51 & 4.37 & 4.98 & 5.41 \\ 
& 11.5 & 3.21 & 3.21 & 3.84 & 3.54 & 4.65 \\ 
& 12 & 3.48 & 3.51 & 3.80 & 3.30 & 3.54 \\ 
& 12.5 & 2.79 & 3.19 & 2.82 & 3.36 & 4.60 \\ 
& 13 & 3.16 & 2.68 & 3.59 & 6.60 & 3.50 \\ 
& 15 & 4.46 & 3.19 & 3.06 & 3.64 & 3.85 \\ 
& 20 & 7.21 & 1.46 & 1.60 & 8.10 & 2.09 \\ 
  \end{tabular}
\end{table}

We can see from the scan, that the transfer factor of the jetty background estimation method becomes larger with larger \pt and smaller with larger \DR. We opt for a transfer factor that is less then unity, and therefore we exclude those choices. Taking into account all factors, we make the choice of $(p,r)=(10\GeV,0.6)$ for muons and $(p,r)=(10\GeV,0.5)$ for electrons.