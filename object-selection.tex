\section{Object definition and selection}
We have seen in~\ref{sec:reconstruction-and-identification} how objects are being reconstructed and identified in our detector. We have also studied the signal signature in~\ref{sec:signal-signature}. In this section we devise an object selection in order to obtain as pure as possible sample of objects in regards to our target leptons, while retaining as much signal as possible. As we have seen in ~\ref{sec:search-strategy}, we are targeting the opposite-charged same-flavor leptons \ellell that result from the \neutt that decays into a \neuto via a \PZstar, \ie, \neuttdecay. In the following section, we choose to present two choices of \gls{dmo}, namely, $\dmo=1.92\GeV$ and $\dmo=5.63\GeV$, \ie, a relatively high \gls{dmo}, and a low one, but not too low as to still be able to have enough electrons surviving the initial reconstruction \pt threshold of $5\GeV$. We also fix the higgsino parameter on $\mu=100\GeV$.

As was the case in~\ref{sec:signal-signature}, the base selection for the following section is requiring at least one jet in the event with $\pt \geq 30\GeV$ and $\abs{\eta}<2.4$. No other selections otherwise. However, unlike~\ref{sec:signal-signature}, we do not weight our objects to any luminosity, as we are interested in the proportion between object types. We differentiate between two types of leptons, ones that originate from our targeted decay \neuttdecay, which will be shown in blue, and those that do not, which we refer to as \emph{other}, and are shown in yellow. Leptons that are marked as resulting from the \neuttdecay decay, which we will refer to as \emph{signal leptons}, are done so by matching a reconstructed lepton to a generator level lepton, which has been checked to have the \neutt as its parent. Lepton marked as \emph{other}, either has been misreconstructed, misidentified or is a result of hadronisation process in a jet (such as the \gls{isr} jet). Our goal here is to select as many blue leptons as possible, while rejecting as many yellow ones as possible. In the following sections, we will refer to \emph{efficiency} as the proportion between the signal leptons passing a selection, divided by the initial number of signal leptons, and to \emph{purity} as the proportion between signal leptons (blue) and the sum of the signal leptons and \emph{other} leptons (yellow). So to rephrase our goal, we are interested in a selection that results in high-efficiency and high-purity. These two quantities can sometimes compete with each other and we have to make compromises.

\subsection{Electrons}

\subsubsection{Signal electron selection}

The electrons have an initial reconstruction \pt threshold of $5\GeV$. The initial working point choice for reconstructed electron is loose (see~\ref{sec:reconstruction-and-identification}). The first distribution we look at in regards to the electrons is their spatial separation from the leading jet in the event, $\DR(\jmath_1,\Pe)$.

\begin{figure}[h]
\centering
\includegraphics[width=0.48\linewidth]{plots/lepton_selection/lepton_selection_dm5p63/none_Electrons_rlj.pdf} \,
\includegraphics[width=0.48\linewidth]{plots/lepton_selection/lepton_selection_dm1p92/none_Electrons_rlj.pdf}  \\
\caption[Spatial seperation between reconstructed electrons and the leading jet $\DR(\jmath_1,\Pe)$]{Spatial seperation between reconstructed electrons with loose ID and the leading jet $\DR(\jmath_1,\Pe)$ for $\dm=5.63\GeV$ (left) and $\dm=1.92\GeV$ (right).}
\label{fig:electrons-dr-lj}
\end{figure}

There are two obvious features we can take from these plots. The first we have explored already in~\ref{sec:signal-signature}, namely, that probing lower \gls{dm} requires access to low \gls{pt} leptons, and since we are limited by a lower threshold of $\pt\geq 5\GeV$ on the electrons, that results in lower signal acceptance as can be seen by the difference between the high \gls{dm} and the low one. The second interesting feature that we can see, is that our signal electrons are located mainly outside of the leading jet. That is because the leading jet is usually an \gls{isr} jet which boosts the \tchiz system to away from it (back-to-back). We therefore make a cut $\DR(\jmath_1,\Pe)>0.4$.

Next we turn into the \gls{pt} distributions. We apply the previous cut of $\DR(\jmath_1,\Pe)>0.4$. As we've already seen in~\ref{sec:muon-eta-pt}, the \gls{pt} distribution depends strongly on \gls{dm}. Even though the distributions in~\ref{sec:muon-eta-pt} were plotted using generator level muons, the electrons distributions follow the same trend. We therefore need to make a choice about which \gls{dm} to favor, \ie, which \gls{dm} we want to be more sensitive to, and we choose the lower \gls{dm} case. Nonetheless we compare the two choices.

\begin{figure}[h]
\centering
\includegraphics[width=0.48\linewidth]{plots/lepton_selection/lepton_selection_dm5p63/none_Electrons_pt.pdf} \,
\includegraphics[width=0.48\linewidth]{plots/lepton_selection/lepton_selection_dm1p92/none_Electrons_pt.pdf}  \\
\caption[\pt distribution of reconstructed electrons with loose ID]{ \pt distribution of reconstructed electrons with loose ID for $\dm=5.63\GeV$ (left) and $\dm=1.92\GeV$ (right). Cut of $\DR(\jmath_1,\Pe)>0.4$ applied.}
\label{fig:electrons-selection-pt}
\end{figure}

We can see, as expected, that the \pt distribution  of the electrons fall more rapidly for the low \dm case. We observe that there are hardly any electrons surviving above $15\GeV$, and therefore we choose to make a cut of $\pt<15\GeV$.

It interesting to look at the $\eta$ distribution after the previous cuts to get a better sense of where most of the non-signal electrons are stil coming from.

\begin{figure}[h]
\centering
\includegraphics[width=0.48\linewidth]{plots/lepton_selection/lepton_selection_dm5p63/none_Electrons_eta.pdf} \,
\includegraphics[width=0.48\linewidth]{plots/lepton_selection/lepton_selection_dm1p92/none_Electrons_eta.pdf}  \\
\caption[\abs{\eta} distribution of reconstructed electrons with loose ID]{ \abs{\eta} distribution of reconstructed electrons with loose ID for $\dm=5.63\GeV$ (left) and $\dm=1.92\GeV$ (right). Cuts of $\DR(\jmath_1,\Pe)>0.4$ and $\pt<15\GeV$ are applied.}
\label{fig:electrons-selection-eta}
\end{figure}

In the case of $\dm=1.92\GeV$,  we can clearly see how worse the endcaps of the \gls{ecal} are performing in comparison with the barrel ($\abs{\eta}<1.48$). The transition is clearly visible through a sharp drop in purity at the transition. It is worse for low-\pt electrons than higher-\pt ones.

We would like to see if requiring a tighter working point for the electron-identification is beneficial. The working point used in the previous distributions is loose. We look turn now to check the effects of requiring either a medium working point, or a tight one. We plot two bins labeled \emph{fail} and \emph{pass}, which correspond to whether the electron passes or failed the identification criteria of a medium or tight working points.

\begin{figure}[h]
\centering
\includegraphics[width=0.48\linewidth]{plots/lepton_selection/lepton_selection_dm5p63/none_Electrons_medium.pdf} \,
\includegraphics[width=0.48\linewidth]{plots/lepton_selection/lepton_selection_dm1p92/none_Electrons_medium.pdf}  \\
\includegraphics[width=0.48\linewidth]{plots/lepton_selection/lepton_selection_dm5p63/none_Electrons_tight.pdf} \,
\includegraphics[width=0.48\linewidth]{plots/lepton_selection/lepton_selection_dm1p92/none_Electrons_tight.pdf}  \\
\caption[medium and tight ID working points distribution of reconstructed electrons]{Medium (top) and tight (bottom) ID working points distributions of reconstructed electrons for $\dm=5.63\GeV$ (left) and $\dm=1.92\GeV$ (right). Cuts of $\DR(\jmath_1,\Pe)>0.4$ and $\pt<15\GeV$ are applied.}
\label{fig:electrons-selection-id}
\end{figure}

To select a medium or a tight working point is equivalent to choosing the relevant right \emph{pass} bin (top for medium, bottom for tight), and rejecting the electrons on the left \emph{fail} bin. We see that although we reject considerable amount of non-signal electrons in the low \dm case by picking either a medium or tight working points, we also loose quite a lot of signal electrons as well. In other words, these selections are very not efficient and will result in low signal acceptance. We therefore decide to use a loose working point for the electrons. We will see that we can still purify the electron selection by relying on isolation instead. We fully discuss and describe our jet-isolation in~\ref{sec:isolation}, but here for the sake of completeness we look at its effect on the purity of the electrons. We compare our custom jet-isolation to the standard definition of lepton isolation, which does not take into account the possibility that two electrons can be produced close to each other (small \DR), as is the case in our signal.

\begin{figure}[h]
\centering
\includegraphics[width=0.48\linewidth]{plots/lepton_selection/lepton_selection_dm5p63/none_Electrons_iso.pdf} \,
\includegraphics[width=0.48\linewidth]{plots/lepton_selection/lepton_selection_dm1p92/none_Electrons_iso.pdf}  \\
\includegraphics[width=0.48\linewidth]{plots/lepton_selection/lepton_selection_dm5p63/none_Electrons_CorrJetNoMultIso11Dr0.5.pdf} \,
\includegraphics[width=0.48\linewidth]{plots/lepton_selection/lepton_selection_dm1p92/none_Electrons_CorrJetNoMultIso11Dr0.5.pdf}  \\
\caption[standard isolation and jet-isolation distribution of reconstructed electrons]{Standard isolation (top) and custom jet-isolation (bottom) distributions of reconstructed electrons with loose ID for $\dm=5.63\GeV$ (left) and $\dm=1.92\GeV$ (right). Cuts of $\DR(\jmath_1,\Pe)>0.4$ and $\pt<15\GeV$ are applied.}
\label{fig:electrons-selection-isolation}
\end{figure}

We observe that the standard lepton isolation does not perform well in terms of efficiency for both \dm cases. In contrast, the custom jet-isolation is performing very well in terms of signal electron efficiency while successfully rejecting considerable amount of non-signal electrons, resulting in a purer sample of electrons.

\subsubsection{Additional electron veto}
\subsubsection{Selection summary}
\subsection{Muons}
\subsubsection{Signal muon selection}
\label{sec:muon-selection}
\subsubsection{Additional muon veto}

\subsection{Missing transverse energy}
\label{subsec:met}

\subsection{Scale factors}
\clearpage

\subsection{Scale factors}

In~\ref{sec:object-selection-electrons} and~\ref{sec:muon-selection} we have studied the selection we apply to electrons and muons respectively. We have therewith made a choice about identification working point. We have used simulation exclusively to draw conclusions about the identification efficiency of the leptons. If we rely on \gls{mc} simulation,  that will produce large systematic errors due to imperfections in modeling both the data and the detector response. We therefore wish to measure the identification efficiency in data, in order to correct the simulation's potentially false efficiency rate. \emph{Efficiency} is defined as how probable it is to reconstruct or identify a lepton. For a lepton $\ell$ the identification efficiency is defined as:

\begin{equation}
\varepsilon_{\ell}^{\mathrm{ID}} = \frac{N_{\ell}(\mathrm{ID})}{N_{\ell}(\mathrm{produced})}
\end{equation}

In \gls{mc} simulation, the number of leptons produced is the same as the number of leptons generated. In data, one must measure it using a data-driven method. Once the efficiencies have been measured both in simulation and in data, a correction factor named \gls{sf} can be applied to the simulation in order to correct for any discrepancies that might arise. The scale factors are defined as the ratio between the efficiency in data to the efficiency in simulation:

\begin{equation}
\mathrm{SF}_{\ell}^{\mathrm{ID}}=\frac{\varepsilon_{\ell}^{\mathrm{ID,Data}}}{\varepsilon_{\ell}^{\mathrm{ID,MC}}},
\end{equation}

dropping the superscript ID we get:

\begin{equation}
\mathrm{SF}_{\ell}=\frac{\varepsilon_{\ell}^{\mathrm{Data}}}{\varepsilon_{\ell}^{\mathrm{MC}}}.
\end{equation}

Once the relevant \gls{sf} have been determined, they are applied for every lepton passing the object selection in the event. The scale factors for loose-ID electrons in our \pt range have been measured centrally by the relevant working group and are applied to the selected electrons. As was determined in~\ref{sec:muon-selection}, our analysis signal muon's lower \pt threshold is $2\GeV$ which is low. Scale factors for medium ID leptons with $\pt\geq 2\GeV$ were computed centrally by the Muon \gls{pog}. The scale factors, however, while matching our muons' \pt range and identification working point, were computed by requiring $\DR > 0.5$  between the muons~\cite{muon-id-sf-2016,muon-id-sf-2016-pres}. As we have seen in~\ref{sec:lepton-dr}, one of are drivers of the sensitivity is the region of $\DR < 0.5$. We would therefore like to validate the scale factors in that region. We would like to show that the efficiencies have no $\DR$ dependence, and in order to do so, we calculate the efficiencies in different $\DR$ regions.

In order to measure such efficiencies in data, one must identify desired leptons with low and easily reducible fakes. A widely used method to perform such a data-driven task is the Tag \& Probe method. In the Tag \& Probe method we examine a mass resonance such as \PZ, \JPsi or \PGU to select particles of the desired type, and probe the efficiency of a particular selection criterion on those particles. The mass resonance will then decay into two same-flavor opposite charged pair of leptons and will form a peak on top of a background. Since we are interested in measuring the efficiency of low \pt muons, we choose to look at dimuon events around the \JPsi mass window. In a dimuon event, we describe one muon as a `tag' and the other as a `probe'. The tag muon is selected with a very tight selection which results in very high certainty that the object corresponds to a real muon produced. The probe is given a loose selection, but since it is constrained to be consistent with a product of a \JPsi, it is almost certain that it originates from a real muon too. Since the shape of the \JPsi is a peak over a background, the background is easily removed by a fit. The probe is then subjected to cuts or constraints which are used to measure a particular efficiency. As stated, in this study we want to show that the efficiency to identify a muon with medium ID working point from a track has no $\DR$ depandance. Therefore, the efficiency we are looking for is defined as:

\begin{equation}
\varepsilon_{\mu}^{\mathrm{ID}} = \frac{N_{\mu}^\mathrm{ID}}{N_{t}}.
\end{equation}

The probe we are using in the denominator is a track passing a loose selection, while the probe track in the numerator is required to match a medium ID working point muon. The number of objects passing a selection is determined by a fit to data and \gls{mc}, to measure the corresponding efficiencies in data and \gls{mc}.

This study is done for year 2016. For the \gls{mc}, we are using the 2016 samples listed in~\ref{sec:sm-mc}. For data, we are  using a single electron trigger, in order for the tagged muon to be independent from the triggered object. The data set is measured to correspond to 36.02\fbinv using the BRIL Work Suite~\cite{bril}. The following trigger paths are used:

\begin{itemize}
\item \texttt{HLT\_Ele27\_WPTight\_Gsf\_v*},
\item \texttt{HLT\_Ele27\_eta2p1\_WPLoose\_Gsf\_v*},
\item \texttt{HLT\_Ele32\_WPTight\_Gsf\_v*},
\item \texttt{HLT\_Ele35\_WPTight\_Gsf\_v*}.
\end{itemize}

We then select an offline loose ID electron with $\pt>27\GeV$. The requirements to select a tag \& probe pair are defined in table~\ref{tab:tag-probe-def}.

\begin{table}[!htb]
	\centering
	\label{tab:tag-probe-def}
		\caption{Selection criteria for Tags and Probes}
		%\vspace{1mm}
			\begin{tabular}{l|l} \hline
			Tag & Probe \\ \hline
			medium ID muon & isolated track\\
			$\pt \geq 5\GeV$ & $2\leq\pt\leq 20\GeV$  ($ \pt\geq 3 \GeV $ for barrel) \\
			$\abs{\eta}<2.4$ & opposite-sign in invariant mass window $[2.5,3.5]\GeV$ \\ \hline
			\end{tabular}
\end{table}

A fit is then performed in an invariant mass window around the \JPsi window of $[2.5,3.5]\GeV$. The signal fit is using a crystal ball function and the continuum is fit with a 6th order polynomial. The fit is repeated twice, where the denominator is done with probe tracks, and the numerator is using medium ID muons that have been matched to said tracks. The \DR~range has been split into 3, and $\abs{\eta}$ of the muons has been split into barrel ($\abs{\eta}<1.2$) and endcaps ($1.2<\abs{\eta}<2.4$). Simulation fits are shown in~\ref{fig:tb-barrel-simulation} for barrel, and~\ref{fig:tb-endcaps-simulation} for endcaps. Data fits are shown in~\ref{fig:tb-barrel-data} for barrel, and~\ref{fig:tb-endcaps-data} for endcaps.

\begin{figure}[!htbp]
\centering
\includegraphics[width=0.32\linewidth]{plots/jpsi_muons_fit_bg_delta_r_single_electron/none_invMass_0_0.3_0_1.2.pdf} \,
\includegraphics[width=0.32\linewidth]{plots/jpsi_muons_fit_bg_delta_r_single_electron/none_invMass_0.3_0.5_0_1.2.pdf}  \,
\includegraphics[width=0.32\linewidth]{plots/jpsi_muons_fit_bg_delta_r_single_electron/none_invMass_0.5_1.5_0_1.2.pdf} \\
\includegraphics[width=0.32\linewidth]{plots/jpsi_muons_fit_bg_delta_r_single_electron/none_id_invMass_0_0.3_0_1.2.pdf} \,
\includegraphics[width=0.32\linewidth]{plots/jpsi_muons_fit_bg_delta_r_single_electron/none_id_invMass_0.3_0.5_0_1.2.pdf}  \,
\includegraphics[width=0.32\linewidth]{plots/jpsi_muons_fit_bg_delta_r_single_electron/none_id_invMass_0.5_1.5_0_1.2.pdf} \\
\caption[Simluation barrel muons fits]{Simluation barrel muons fits for denominator (top) and numerator (bottom) for $0<\DR<0.3$  (left), $0.3<\DR<0.5$ (center), $0.5<\DR<1.5$ (right)}
\label{fig:tb-barrel-simulation}
\end{figure}

\begin{figure}[!htbp]
\centering
\includegraphics[width=0.32\linewidth]{plots/jpsi_muons_fit_bg_delta_r_single_electron/none_invMass_0_0.3_1.2_2.4.pdf} \,
\includegraphics[width=0.32\linewidth]{plots/jpsi_muons_fit_bg_delta_r_single_electron/none_invMass_0.3_0.5_1.2_2.4.pdf} \,
\includegraphics[width=0.32\linewidth]{plots/jpsi_muons_fit_bg_delta_r_single_electron/none_invMass_0.5_1.5_1.2_2.4.pdf} \\
\includegraphics[width=0.32\linewidth]{plots/jpsi_muons_fit_bg_delta_r_single_electron/none_id_invMass_0_0.3_1.2_2.4.pdf} \,
\includegraphics[width=0.32\linewidth]{plots/jpsi_muons_fit_bg_delta_r_single_electron/none_id_invMass_0.3_0.5_1.2_2.4.pdf} \,
\includegraphics[width=0.32\linewidth]{plots/jpsi_muons_fit_bg_delta_r_single_electron/none_id_invMass_0.5_1.5_1.2_2.4.pdf}  \\
\caption[Simluation endcaps muons fits]{Simluation endcaps muons fits for denominator (top) and numerator (bottom) for $0<\DR<0.3$  (left), $0.3<\DR<0.5$ (center), $0.5<\DR<1.5$ (right)}
\label{fig:tb-endcaps-simulation}
\end{figure}

\begin{figure}[!htbp]
\centering
\includegraphics[width=0.32\linewidth]{plots/jpsi_muons_fit_data_delta_r_single_electron/none_invMass_0_0.3_0_1.2.pdf} \,
\includegraphics[width=0.32\linewidth]{plots/jpsi_muons_fit_data_delta_r_single_electron/none_invMass_0.3_0.5_0_1.2.pdf}  \,
\includegraphics[width=0.32\linewidth]{plots/jpsi_muons_fit_data_delta_r_single_electron/none_invMass_0.5_1.5_0_1.2.pdf} \\
\includegraphics[width=0.32\linewidth]{plots/jpsi_muons_fit_data_delta_r_single_electron/none_id_invMass_0_0.3_0_1.2.pdf} \,
\includegraphics[width=0.32\linewidth]{plots/jpsi_muons_fit_data_delta_r_single_electron/none_id_invMass_0.3_0.5_0_1.2.pdf}  \,
\includegraphics[width=0.32\linewidth]{plots/jpsi_muons_fit_data_delta_r_single_electron/none_id_invMass_0.5_1.5_0_1.2.pdf} \\
\caption[Data barrel muons fits]{Data barrel muons fits for denominator (top) and numerator (bottom) for $0<\DR<0.3$  (left), $0.3<\DR<0.5$ (center), $0.5<\DR<1.5$ (right)}
\label{fig:tb-barrel-data}
\end{figure}

\begin{figure}[!htbp]
\centering
\includegraphics[width=0.32\linewidth]{plots/jpsi_muons_fit_data_delta_r_single_electron/none_invMass_0_0.3_1.2_2.4.pdf} \,
\includegraphics[width=0.32\linewidth]{plots/jpsi_muons_fit_data_delta_r_single_electron/none_invMass_0.3_0.5_1.2_2.4.pdf} \,
\includegraphics[width=0.32\linewidth]{plots/jpsi_muons_fit_data_delta_r_single_electron/none_invMass_0.5_1.5_1.2_2.4.pdf} \\
\includegraphics[width=0.32\linewidth]{plots/jpsi_muons_fit_data_delta_r_single_electron/none_id_invMass_0_0.3_1.2_2.4.pdf} \,
\includegraphics[width=0.32\linewidth]{plots/jpsi_muons_fit_data_delta_r_single_electron/none_id_invMass_0.3_0.5_1.2_2.4.pdf} \,
\includegraphics[width=0.32\linewidth]{plots/jpsi_muons_fit_data_delta_r_single_electron/none_id_invMass_0.5_1.5_1.2_2.4.pdf}  \\
\caption[Data endcaps muons fits]{Data endcaps muons fits for denominator (top) and numerator (bottom) for $0<\DR<0.3$  (left), $0.3<\DR<0.5$ (center), $0.5<\DR<1.5$ (right)}
\label{fig:tb-endcaps-data}
\end{figure}

The efficiencies and corresponing scale factors can be seen in ~\ref{fig:tb-eff-sf}. The scale factors are statistically consistent with unity, and show no discernible $\DR$ dependence. A similar study has been carried out with simulation and data for 2017 and 2018 in~\cite{muon-id-sf-2017-8} and did not observe a $\DR$ dependence either. The recommendation from the \gls{pog} as a results of these studies are to use the calculated scale factors provided by them with an additional systematic of 1\% for muons with $\pt<20\GeV$.

\begin{figure}[!htbp]
\centering
\includegraphics[width=0.48\linewidth]{plots/scale_factors/barrelDeltaRSingleElectron.pdf} \,
\includegraphics[width=0.48\linewidth]{plots/scale_factors/endcapsDeltaRSingleElectron.pdf}  \\
\includegraphics[width=0.48\linewidth]{plots/scale_factors/barrelDeltaRisoScaleFactorsSingleElectron.pdf} \,
\includegraphics[width=0.48\linewidth]{plots/scale_factors/endcapsDeltaRisoScaleFactorsSingleElectron.pdf} \\
\caption[Efficiencies and scale factors]{Efficiencies (top) and scale factors (bottom) for barrel muons (left) and endcaps muons (right).}
\label{fig:tb-eff-sf}
\end{figure}

\subsection{Tracks and multivariate selection }
\subsection{Isolation}
\label{sec:isolation}
