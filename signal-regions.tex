\clearpage
\section{Optimization of search bins}
\label{sec:signal-regions}

The signal region is split into various search bins in the range of the event BDT classifier output greater than zero. The final likelihood fit is performed using all of the bins simultaneously, and this approach is a type of \emph{shape analysis}. The significance is computed in each bin, and the individual significance values are then combined to yield a single significance value for a given signal hypothesis.

As a general rule of thumb, the signal purity increases as a function of the BDT output score. This means that the most significant bin is likely to be to the right end of the distribution. Finding an ideal choice of bin boundaries can be challenging because the distributions are not smooth, but are made up of event counts with statistical fluctuations. The first step in defining the \glspl{sr} is defining the rightmost division that becomes the left edge of the most sensitive bin, stretching all the way up to the maximum BDT output value of 1. To choose this bin boundary, a scan is performed over all possible lower thresholds on the BDT score in the considered range, employing a step size $\varepsilon$. In each step $i$, a significance is computed for a bin of size $i\cdot \varepsilon$,\ie, in the interval $\left[ 1-i\cdot \varepsilon, 1 \right]$. One can then pick the left bin by taking the maximum of the series of values resulting in the previous step.

Three open points regarding the binning optimisation warrant further elaboration. The first is the choice of measure for estimating the significance. Since the final significance, combination, and exclusion limit are calculated using the CLs method with asymptotic limits, and is somewhat intractable for a study like this~\cite{higgs-combine-site}, a simple estimate is employed at this stage, which is reviewed in~\cite{pvalue,Cousins:2007bmb} and referred to as the Z-value. The Z-value is related to the $p$-value by specifying the corresponding number of standard deviations in a one-tailed test of a Gaussian (normal) variate:
\begin{equation}
Z = \Phi^{-1}(1-p)=-\Phi^{-1}(p).
\end{equation}
Given the number of signal events count $\hat{s}$, background events count $\hat{b}$ and its corresponding error $\delta \hat{b}$, an estimator for the significance is given by
\begin{equation}
Z = \frac{\hat{s}}{\sqrt{\hat{b}+\delta \hat{b}^2}}.
\end{equation}
The background event count is estimated using the data-driven methods described in~\ref{sec:background-estimation}. They all involve counting events in a sideband and multiplying them be a transfer factor computed in a control region:
\begin{equation}
\hat{b} = \mathrm{N}^{\mathrm{SR}}_{\text{sideband}}\cdot \mathrm{TF},
\end{equation}
where the transfer factor TF is given by 
\begin{equation}
\mathrm{TF}=\frac{\mathrm{N}^{\mathrm{CR}}_{\text{main band}}}{\mathrm{N}^{\mathrm{CR}}_{\text{sideband}}}.
\end{equation}
The error propagation formula yields
\begin{equation}
\left( \frac{\delta \hat{b}}{\hat{b}}  \right)^2=\left( \frac{\delta \mathrm{N}^{\mathrm{SR}}_{\text{sideband}}}{\mathrm{N}^{\mathrm{SR}}_{\text{sideband}}}  \right)^2 + \left( \frac{\delta \mathrm{TF}}{\mathrm{TF}}  \right)^2,
\end{equation}
which results in
\begin{equation}
\delta \hat{b}^2=\hat{b}^2 \left[ \left( \frac{\delta \mathrm{N}^{\mathrm{SR}}_{\text{sideband}}}{\mathrm{N}^{\mathrm{SR}}_{\text{sideband}}}  \right)^2 + \left( \frac{\delta \mathrm{TF}}{\mathrm{TF}}  \right)^2 \right] .
\end{equation}

The second point that needs to be addressed is the choice of which signal point or points to optimize. Each model point yields a different signal event count $\hat{s}$ and therefore produces different significance values. To select the optimal bin boundaries, a range of signal model points is considered along the edge of the exclusion limit and thus yield to the strongest limit contours.

The third and final point concerns the choice of step size $\varepsilon$. If $\varepsilon$ is too small, there will be steps where no events are encountered in either the signal or the background due to the finite statistics. Therefore, encountered background event causes a discrete jump in the significance, an artificial effect that can lead to overtraining. It will produce meaningfully different results given a statistically independent set of events. To avoid overtraining, a relatively large step size of $\varepsilon=0.05$ was chosen to balance the need for sufficient statistics for all steps in the scan with the benefits of higher granularity.

After the most significant bin has been fixed, the remaining BDT range from 0 to the low edge of the tightest bin is divided equally in order to increase sensitivity, particularly to models with small $\dm$. For reason of statistics, for the dimuon category, these bin widths are chosen as 0.1, while for the exclusive track categories, it is 0.05. The final signal regions are listed in Table~\ref{tab:signal-regions}.

\begin{table}[hp]
	\centering
	
		\caption{\label{tab:signal-regions}Signal Regions}
		%\vspace{1mm}
			\begin{tabular}{lcccc} \hline
			Category & Flavor & Phase & SR & Signal Regions \\ \hline
			Dilepton & Muon & all &  6 & $[0,0.1,0.2,0.3,0.4,0.5,1]$ \\
			
			Exclusive Track & Muon & 0 & 13 & $[0,0.05,0.1,0.15,0.2,\cdots,0.5,0.55,0.6,1]$ \\ 
			Exclusive Track & Muon & 1 & 12 & $[0,0.05,0.1,0.15,0.2,\cdots,0.5,0.55,1]$ \\	
			Exclusive Track & Electron & all & 11 & $[0,0.05,0.1,0.15,0.2,\cdots,0.5,1]$ \\			
			
			\hline
			\end{tabular}
\end{table}
