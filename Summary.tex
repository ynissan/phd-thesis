\chapter{Summary}
\label{sec:summary}

This thesis presented a search for compressed mass Higgsino production with low-momentum lepton tracks with the CMS experiment. It attempts to either find a signature, which would mean that the search for a dark matter candidate has finally came to an end, or to expend previous exclusion limits. The dark matter candidate is suggested to be a Higgsino neutralino, and is interpenetrated in this search as part of a simplified model. Such a model is well motivated not only because it provides a suitable WIMP dark matter candidate, but also because it is realized as natural SUSY extension to the SM.

The analysis presented here attempts to use the full run 2 data collected by CMS between the data collecting years of 2016-2018. Therefore, handle has been care in order to study the two phases of the tracker and the challenges that arise in each data taking period. Most of the background estimation is done using a data-driven method. Only the minor isolated background uses simulation. The search uses a unique phase space of muons with low transverse momentum that are close to each other, which has not been explored by previous searches. It is that unique phase space that provides most of the sensitivity of this search. In addition,  track plus lepton categories were introduced in this search to expand the limits a little further. In order to be able to use such low momentum muons, a study comparing the identification efficiency between simulation and data has been performed to compute scale factors. Dedicated isolation method has been used, which optimally selects signal leptons, that are often nearly collinear and spoil each others' standard isolation. In addition, multivariate discriminants were used to enhance sensitivity .

This search is in the last stages of approval by the relevant committee at CMS. Therefore, only a partial unblinding of 10\% of the data was allowed. In that fraction of the data, no new physics was observed, but also no limits have been able to be set. This picture is expected to change once the full data collected in run 2 is allowed to be unblinded. In order to appreciate the potential of the search, expected limits have been computed for the full run 2 luminosity. The computed expected limits show that this search is able to probe lower in \dm than previous searches with similar final states.

In conclusion, the search demonstrates its ability to expand on previous limits and probe into more compressed scenarios than before. Hopefully with the soon-to-be fully unblinded data, either a discovery would be made, or the expected limits will be realized in full.