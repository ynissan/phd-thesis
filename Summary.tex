\chapter{Summary}
\label{sec:summary}

This thesis presents a search for compressed mass Higgsino production with low-momentum lepton tracks using the CMS experiment. The goal is to either discover a signature, indicating the end of the search for a dark matter candidate, or to expand upon previous exclusion limits. The proposed dark matter candidate is a Higgsino neutralino, which is interpreted within this search as part of a simplified model. This model is highly motivated as it not only offers a suitable WIMP dark matter candidate but also serves as a natural extension of the SM through SUSY.

The analysis presented in this work aims to utilize the complete Run 2 dataset collected by CMS during the data-taking years of 2016-2018. Therefore, special attention has been given to studying the two phases of the tracker and the challenges associated with each data-taking period. The background estimation method predominantly relies on a data-driven approach, with simulation only being utilized for minor isolated background contributions. This search focuses on a unique phase space characterized by low transverse momentum muons that are in proximity to each other, an unexplored region in previous searches. It is within this distinctive phase space that the search achieves its highest sensitivity. Moreover, the inclusion of track plus lepton categories in this search extends the limits of the analysis. To account for the usage of low-momentum muons, a comprehensive study comparing the identification efficiency between simulation and data has been conducted to derive scale factors. Additionally, a dedicated isolation method has been employed, which optimally selects signal leptons that are often nearly collinear and consequently affect each other's standard isolation. Furthermore, the utilization of multivariate discriminants enhances the sensitivity of the analysis.

This search is currently in the final stages of approval by the relevant committee at CMS. As a result, only a partial unblinding of 10\% of the data was permitted. In this fraction of the data, no new physics phenomena were observed, and no definitive limits could be set. However, it is anticipated that this situation will change once the full dataset collected during Run 2 is unblinded. To gauge the potential of the search, expected limits have been calculated for the entire run 2 luminosity. The computed expected limits demonstrate that this search has the capability to explore lower values of \dm compared to previous searches with similar final states.

In conclusion, this search showcases its capability to extend beyond previous limits and explore more compressed scenarios. With the forthcoming unblinding of the complete dataset, there is hope that either a significant discovery will be made or the expected limits will be fully realized.