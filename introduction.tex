\chapter{Introduction}

In 2012, the last piece of the \gls{sm} was discovered at the \gls{lhc}: the Higgs boson. However, that does not mean that all mysteries and puzzles were solved. It is known that the SM is flawed, and many extensions attempting to solve known issues with the SM have been proposed. These theories are referred to as \gls{bsm}. One very popular BSM theory is \gls{susy}. Some models of SUSY, such as the one considered in the analysis presented in this thesis, provide solutions to some of the problems with the SM, such as the existence of a \gls{dam} candidate and a natural explanation for the hierarchy problem.

One of the leading candidates for a \gls{dam} particle is a \gls{wimp}. \gls{susy} with Higgsino-dominated neutralino as \gls{lsp} provides such a \gls{wimp} candidate. In addition, when the mass splitting between the LSP and a heavier chargino or neutralino is small, such a model describes a natural realization of \gls{susy}. These so-called compressed SUSY scenarios can be challenging to probe due to the low momentum of the visible decay products. This search targets compressed scenarios where the mass difference between the two lightest neutralinos is between $1-5\GeV$. The search considers low-momentum muons that are very close to each other. Since this is a challenging final state, dedicated isolation criterion is developed, and scale factors are studies to compare identification efficiency differences between data and simulation. In addition, machine learning techniques are employed. Data-driven methods are used to estimate the majority of the background processes. The goal of the thesis is to probe the unexplored region of more compressed scenarios not covered by previous analyses.

In Chapter~\ref{sec:theory} of the thesis, the theoretical background is introduced. The target model is motivated, and the shortcomings of the SM are explained. In Chapter~\ref{sec:experimental-setup}, the experimental setup is laid out. In Chapter~\ref{sec:search} the search is explored in great detail. At the end of the chapter, the results of both the partial unblinding and the expected limits of the full luminosity are given. The thesis concludes in the summary in Chapter~\ref{sec:summary}.