\chapter{Experimental setup}

One of the most useful methods to study the subatomic world of particle physics is by using particle accelerators. In such machines, particles are accelerated to very high speeds and energies, and smashed into each other. The resulted particles are then studied by measuring them in a detector. At the time of writing this thesis, the largest currently active accelerator is the \gls{lhc} near Geneva, Switzerland, operated by the \gls{cern}. For the analysis presented in this thesis, data from the \gls{cms} experiment has been analyzed. In this chapter, the \gls{lhc} is described in~\ref{sec:lhc},  while the \gls{cms} experiment is described in~\ref{sec:cms}.

\section{The Large Hadron Collider}
\label{sec:lhc}

\section{The Compact Muon Solenoid experiment}
\label{sec:cms}



\section{Simulation of events}
