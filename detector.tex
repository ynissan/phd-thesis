\chapter{Experimental setup}

One of the most useful methods to study the subatomic world of particle physics uses particle colliders. In such machines, particles are accelerated to very high speeds and energies, and smashed into each other. The particles that emerge from the collisions are then measured in a particle detector and then studied and analyzed. At the time of writing this thesis, the largest and most high energy collider to date is the \gls{lhc} located in Geneva, Switzerland, operated by the \gls{cern}. For the present work, data from the \gls{cms} experiment has been analyzed. In this chapter, the \gls{lhc} is described in~\ref{sec:lhc},  while the \gls{cms} experiment is described in~\ref{sec:cms}.

\section{The Large Hadron Collider}
\label{sec:lhc}

\section{The Compact Muon Solenoid experiment}
\label{sec:cms}



\section{Simulation of events}
