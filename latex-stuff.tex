\chapter{Latex stuff}

\section{Some examples}

\subsection{Multiline comment}
%% This is a multiline comment
\comment{
This is a comment
}
This is a line in introduction.
\subsection{Fixme note}
This is the introduction to the thesis. \fxnote{This is a fixme note}
\fxnote[inline]{what}
\fxnote[inline=true]{WHAT THE HELL}
AFTER

\subsection{Tables}

h - here
t - top
b - bottom
p - special page
! - even if not pretty 

\begin{table}[hp]
	\centering
	\label{tab:mytab}
		\caption{Table captions are above the table whereas figure captions are below.}
		%\vspace{1mm}
			\begin{tabular}{lcc} \hline
			Parameter & Value 1 & Value 2 \\ \hline
			$s$ & 10.0 & 20.0 \\
			$t$ & 20.0 & 30.0 \\
			$u$ & 30.0 & 40.0 \\ \hline
			\end{tabular}
\end{table}

\subsection{Cross References}
\label{subsec:marker}

\ref{subsec:marker}
\pageref{subsec:marker}
section~\ref{subsec:marker}

\subsection{Particles}
Hello World $\PSGczDo \PGp \PGhc \GeV \ETm$ hey \GeV \ETm \PGp
new one \neuto
 \PSGczDo
\subsection{Citing}
\cite{sos} SOS analysis 
\subsection{Glossary}
Using glossary for compter \gls{symb:computer} plural form \glspl{symb:computer}
upper case first \Gls{symb:computer} upper case first plural \Glspl{symb:computer}. To use for symbol \glssymbol{symb:pi}

\subsection{Acronyms}
First use of acronym \gls{sos} and second \gls{sos}. 
You can reset this and do again \glsreset{sos} \gls{sos} and second time again \gls{sos}. 
Long version \acrlong{sos}.
Full version \acrfull{sos}.
Short version \acrshort{sos}.
