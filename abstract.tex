\chapter{Abstract}

A search for neutral Higgsino particles in final states with large missing transverse momentum and either a lepton-lepton pair or lepton-track pair is presented. The signal scenario considers neutralinos comprising two mass eigenstates differing in mass by small values of approximately 1-5 $\GeV$, where the heavier neutralino decays into the lighter neutralino and two same-flavor leptons. The leptons possess small transverse momentum and thus often fail to be reconstructed. To recover sensitivity, events in which only one of the leptons is identified, while the second lepton is measured as a simple track with the opposite charge, are considered, complementing the case where both leptons are reconstructed. A dedicated isolation method is used, that optimally selects signal leptons, which are often nearly collinear and spoil each other's standard isolation. The custom isolation variable serves to define control regions used to estimate standard model backgrounds for the search. Multivariate discriminants are used to enhance sensitivity at various stages of the selection. The search does not overlap with previous searches in the dilepton final state and probes signal model phase space that has not been explored by previous searches.

The search is designed to analyze the proton-proton collision data collected with the CMS experiment during Run 2 with luminosity of $137\fbinv$ at a center-of-mass energy of $\sqrt{s}=13\TeV$. The interpretation is done with a simplified model of compressed mass Higgsinos. The full luminosity is used to calculate the background yield from data and the expected limits. Near the LEP limits of $m_\PSGcpmDo\approx 100\GeV$, $\dmpm$ ($\dmo$) down to $0.8\GeV$ ($1.6\GeV$) is expected to be excluded. At higher mass splittings between 2 and $2.5\GeV$, a chargino mass of up to almost $160\GeV$ is expected to be excluded. The expected limits improve upon previous searches with similar final states. For the purpose of this thesis, 10\% of the data was unblinded showing good agreement between the measured data and the predicted background from Standard Model processes.

\chapter{Zusammenfassung}

Es wird eine Suche nach neutralen Higgsino-Teilchen in Endzuständen mit großer fehlender transversaler Impuls und entweder einem Leptonen-Leptonen-Paar oder einem Lepton-Spur-Paar präsentiert. Das Signal-Szenario betrachtet Neutralinos, die aus zwei Masseneigenzuständen bestehen, die sich in der Masse um kleine Werte von ungefähr 1-5 $\GeV$ unterscheiden, wobei das schwerere Neutralino in das leichtere Neutralino und zwei Leptonen mit gleicher Flavour zerfällt. Die Leptonen besitzen einen kleinen transversalen Impuls und werden daher oft nicht rekonstruiert. Um die Sensitivität wiederherzustellen, werden Ereignisse betrachtet, bei denen nur eines der Leptonen identifiziert wird, während das zweite Lepton als einfache Spur mit entgegengesetzter Ladung gemessen wird. Dies ergänzt den Fall, in dem beide Leptonen rekonstruiert werden. Es wird eine spezielle Isolationsmethode verwendet, die optimal Signal-Leptonen auswählt, die oft nahezu kollinear sind und sich gegenseitig in ihrer standardmäßigen Isolation stören. Die benutzerdefinierte Isolationsvariable dient dazu, Kontrollbereiche zu definieren, die zur Abschätzung der Hintergrundereignisse des Standardmodells für die Suche verwendet werden. Multivariate Diskriminanten werden verwendet, um die Sensitivität in verschiedenen Stadien der Selektion zu erhöhen. Die Suche überschneidet sich nicht mit früheren Suchen im Dilepton-Endzustand und untersucht Signalmodell-Phasenraum, der von früheren Suchen noch nicht erforscht wurde.

Die Suche ist darauf ausgelegt, die mit dem CMS-Experiment während Run 2 bei einer Luminosität von $137\fbinv$ bei einer Schwerpunktsenergie von $\sqrt{s}=13\TeV$ gesammelten Proton-Proton-Kollisionsdaten zu analysieren. Die Interpretation erfolgt mit einem vereinfachten Modell von komprimierten Higgsinos. Die volle Luminosität wird verwendet, um die Hintergrundereigniszahl aus den Daten und die erwarteten Grenzwerte zu berechnen. Nahe den LEP-Grenzwerten von $m_\PSGcpmDo\approx 100\GeV$ wird erwartet, dass $\dmpm$ ($\dmo$) bis hinunter zu $0.8\GeV$ ($1.6\GeV$) ausgeschlossen wird. Bei größeren Massendifferenzen zwischen 2 und $2.5\GeV$ wird erwartet, dass eine Chargino-Masse von bis zu fast $160\GeV$ ausgeschlossen wird. Die erwarteten Grenzwerte verbessern die Ergebnisse früherer Suchen mit ähnlichen Endzuständen. Für diese Arbeit wurden 10\% der Daten unverblindet, wobei eine gute Übereinstimmung zwischen den gemessenen Daten und dem vorhergesagten Hintergrund aus Standardmodell-Prozessen festgestellt wurde.