\chapter{Abstract}

A search for neutral Higgsino particles in final states with large missing transverse momentum and either a lepton-lepton pair or lepton-track pair is presented. The signal scenario considers neutralinos comprising two mass eigenstates differing in mass by small values of approximately 1-5 $\GeV$, where the heavier neutralino decays into the lighter neutralino and two same-flavor leptons. The leptons possess small transverse momentum and thus often fail to be reconstructed. To recover sensitivity, events in which only one of the leptons is reconstructed while the second lepton is identified as a simple track with opposite charge are considered, complementing the case where both leptons are reconstructed. Dedicated isolation method is used, that optimally selects signal leptons, which are often nearly collinear and spoil each others' standard isolation. The custom isolation variable serves to define control regions used to estimate standard model backgrounds to the search. Multivariate discriminants are used to enhance sensitivity at various stages of the selection. The search is non-overlapping with previous searches in the dilepton final state, and probes signal model phase space not explored by previous searches. 

The search is designed to analyze the proton-proton collision data collected with the CMS experiment during Run 2 with luminosity of $137\fbinv$ at center of mass energy of $\sqrt{s}=13\TeV$. The interpretation is done with a simplified model of compressed mass Higgsinos. The analysis is at the last stages of approval for full unblinding, and for the purpose of this thesis, 10\% of the data was unblinded. No new physics has been found, but also no observed limit was able to be set. The full luminosity's potential is realized through the calculation of the expected limits. Near the LEP limits of $m_\PSGcpmDo\approx 100\GeV$, $\dmpm$ ($\dmo$) down to $0.8\GeV$ ($1.6\GeV$) is expected to be excluded. At higher mass splittings between 2 and $2.5\GeV$, chargino mass of up to almost $160\GeV$ is expected to be excluded. The expected limits improve on previous searches with similar final states.