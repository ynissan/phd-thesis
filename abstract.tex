\chapter{Abstract}

A search for neutral Higgsino particles in final states with large missing transverse momentum and either a lepton-lepton pair or lepton-track pair is presented. The signal scenario considers neutralinos comprising two mass eigenstates differing in mass by small values of approximately 1-5 $\GeV$, where the heavier neutralino decays into the lighter neutralino and two same-flavor leptons. The leptons possess small transverse momentum and thus often fail to be reconstructed. To recover sensitivity, events in which only one of the leptons is identified, while the second lepton is measured as a simple track with the opposite charge, are considered, complementing the case where both leptons are reconstructed. A dedicated isolation method is used, that optimally selects signal leptons, which are often nearly collinear and spoil each other's standard isolation. The custom isolation variable serves to define control regions used to estimate standard model backgrounds for the search. Multivariate discriminants are used to enhance sensitivity at various stages of the selection. The search does not overlap with previous searches in the dilepton final state and probes signal model phase space that has not been explored by previous searches.

The search is designed to analyze the proton-proton collision data collected with the CMS experiment during Run 2 with luminosity of $137\fbinv$ at a center-of-mass energy of $\sqrt{s}=13\TeV$. The interpretation is done with a simplified model of compressed mass Higgsinos. The full luminosity is used to calculate the background yield from data and the expected limits. Near the LEP limits of $m_\PSGcpmDo\approx 100\GeV$, $\dmpm$ ($\dmo$) down to $0.8\GeV$ ($1.6\GeV$) is expected to be excluded. At higher mass splittings between 2 and $2.5\GeV$, a chargino mass of up to almost $160\GeV$ is expected to be excluded. The expected limits improve upon previous searches with similar final states. For the purpose of this thesis, 10\% of the data was unblinded showing good agreement between the measured data and the predicted background from Standard Model processes.

\chapter{Zusammenfassung}

In dieser Arbeit wird eine Suche nach neutralen Higgsino-Teilchen in Endzuständen mit einem großen Betrag an fehlendem Transversalimpuls und entweder einem Lepton-Lepton-Paar oder einem Lepton-Spur-Paar präsentiert. Das Signal-Szenario ist definiert durch Neutralinos, die zwei Masseneigenzuständen mit kleinen Massendifferenzen von ungefähr 1-5 $\GeV$ entsprechen. Das schwerere Neutralino zerfällt dabei in das leichtere Neutralino und zwei Leptonen mit gleichem Flavour. Die Leptonen haben wenig Transversalimpuls und werden daher oft nicht vollständig rekonstruiert. Um trotzdem Sensitivität zu erlangen, werden Ereignisse betrachtet, bei denen nur eines der Leptonen identifiziert wird, während das zweite Lepton im Detektor nur als einfache Spur mit entgegengesetzter Ladung gemessen wird. Dies ergänzt den Fall, in dem beide Leptonen vollständig rekonstruiert werden. Es wird eine spezielle Methode zur Berechnung der Isolation verwendet, die optimal auf Signal-Leptonen abgestimmt ist, da diese oft nahezu kollinear sind und sich gegenseitig in ihrer standardmäßigen Isolation stören. Die spezielle Isolationsvariable wird dazu verwendet, Kontrollbereiche zu definieren, die zur Abschätzung der Anzahl an Untergrundereignisse für die Suche benutzt werden. Multivariate Analysemethoden werden verwendet, um die Sensitivität in verschiedenen Selektionsschritten zu erhöhen. Die Suche überschneidet sich nicht mit früheren Di-Lepton-Suchen und untersucht Phasenraumbereiche des Signal-Modells, die von früheren Suchen noch nicht erforscht wurden.

Die Suche ist darauf ausgelegt, die mit dem CMS-Experiment während Run 2 bei einer Schwerpunktsenergie von $\sqrt{s}=13\TeV$ gesammelten Proton-Proton-Kollisionsdaten zu analysieren. Diese Daten entsprechen einer integrierten Luminosität von $137\fbinv$. Die Interpretation erfolgt mithilfe eines vereinfachten Modells, das durch Higgsinos mit kleinen Massendifferenzen definiert ist. Die volle Luminosität wird verwendet, um die Anzahl der erwarteten Untergrundereignisse mithilfe der Daten sowie die erwarteten Ausschlussgrenzen zu berechnen. Nahe der LEP-Ausschlussgrenzen von\\ ${m_\PSGcpmDo\approx 100\GeV}$ wird erwartet, dass Modelle mit $\dmpm$ ($\dmo$) größer als $0,8\GeV$ ($1,6\GeV$) ausgeschlossen werden können. Bei größeren Massendifferenzen zwischen 2 und $2,5\GeV$ wird erwartet, dass Charginos mit Massen von bis zu ungefähr $160\GeV$ ausgeschlossen werden können. Die erwarteten Ausschlussgrenzen verbessern die Ergebnisse früherer Suchen mit ähnlichen Endzuständen. Für die beobachteten Ausschlussgrenzen wurden 10\% der Daten benutzt, wobei eine gute Übereinstimmung zwischen den aufgenommenen Daten und den vorhergesagten Untergrundereignissen aus Standardmodell-Prozessen festgestellt wurde.