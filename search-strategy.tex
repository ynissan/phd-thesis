\clearpage
\section{Search strategy}
\label{sec:search-strategy}

The search details are described in depth in the upcoming sections. However, it is useful to have a very quick overview of the strategy so that it is easier to follow. This analysis targets the two leptons resulting from the decay of the \neutt. Those are opposite-charge, same-flavor leptons \ellell resulting from the \neutt that decays into a \neuto and a \PZstar, \ie, \neuttdecay. If there is a \chargino present, such as in the production of \tchiwz, it is assumed to either decay hadronically or that the resulting lepton is not identified. The invariant mass of the two leptons resulting from the decay has a unique shape due to the limited allowed phase space of the 3-body decay and is restricted to the mass difference between \neutt and \neuto, that is, \dm. Therefore, the $\mll$ distribution is expected to have an edge at \dm. The presence of two \neuto in the final state leads to high \MET in the event, which leads to the use of triggers based on missing transverse energy. Typical for such topographies, an ISR jet is also required for increased sensitivity.

The analysis has three channels corresponding to two physical final states: two identified muons, one identified muon and one track, and one identified electron and one track. The channels with one identified lepton and one track result from one of the leptons not being identified due to the low identification efficiency for low momentum objects. There is no channel corresponding to two identified electrons, as it has already been analyzed in the SOS paper~\cite{sos}. All channels utilize a BDT to select the signal and reject SM background. The BDT discriminator outputs are also used to define the signal regions for signal extraction and limit settings, as well as control regions for background estimation purposes. The identified lepton plus track channels utilize an additional object-level BDT discriminator (track picking BDT) to select a track among the collection of tracks in an event, which hopefully corresponds to the non-identified lepton. The key points of each channel are summarized in the sections below.


\subsection{Final state with two identified muons}
\label{sec:dimuon-category}

\begin{itemize}
\item \textbf{Defining objects:} two identified opposite-charge muons.
\item \textbf{Signal regions:} using bins in an event BDT score of $\text{BDT}>0$.
\item \textbf{Background estimation:} isolated background arising from leptonic decays of $\tau\tau$ is estimated using MC, and non-isolated background is estimated using dedicated isolation sideband CR in data.

\end{itemize}

\subsection{Final state with one identified lepton and one track}
\label{sec:exclusive-track-category}

\begin{itemize}
\item \textbf{Defining objects:} one identified lepton (muon or electron) and one opposite charge track having maximum track picking BDT score.
\item \textbf{Signal regions:} using bins in an event BDT score of $\text{BDT}>0$.
\item \textbf{Background estimation:} using same-sign CR in data.
\end{itemize}