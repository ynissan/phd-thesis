\clearpage
\section{Search strategy}
\label{sec:search-strategy}

The invariant mass of the two leptons resulting from the decay of the \neutt has a unique shape due to the limited allowed phase space of the 3-body decay. As the \neutt decays into \neuto and \ellell through a \PZstar, the allowed phase space of the dilepton pair is restricted to the mass difference between \neutt and \neuto, that is, \dm. Therefore, the $\mll$ distribution is expected to have an edge at \dm. 

the focus is on selecting opposite-charge, same-flavor leptons \ellell resulting from the \neutt that decays into a \neuto and a \PZstar, \ie, \neuttdecay. Two choices of $\dmo$ are presented in the following section: a relatively high $\dmo$ of $\dmo=5.63\GeV$ and a low $\dmo$ of $\dmo=1.92\GeV$, but not so low as to prevent enough electrons from surviving the initial reconstruction \pt threshold of $5\GeV$. The higgsino parameter is fixed at $\mu=100\GeV$.

\subsection{Final state with two muons}
\label{sec:dimuon-category}

\subsection{Final state with one lepton and one track}
\label{sec:exclusive-track-category}

